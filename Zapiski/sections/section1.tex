\section{Convexity}

\subsection{Locally convex spaces}

Let $\F \in \{\R, \C\}$ be a field.

\begin{definition}
    A \emph{topological vector space} (TVS) is a $\F$-vector space that is also a topological space
    and the two structures are compatible. This means that the usual operations on vector spaces
    $$V \times V \to v,\ (x, y)\mapsto x + y,\qquad \F \times V \to V,\ (\lambda, x)\mapsto \lambda x$$
    are continuous maps.
\end{definition}

\begin{example}
    Normed spaces are TVS.
\end{example}

\begin{definition}
    Let $V$ be a $\F$-space. Map $p: V \to \R$ is a \emph{seminorm} if:
    \begin{enumerate}
        \item $p(x) \geq 0,\ \forall x \in V$ (positivity);
        \item $p(\lambda x) = |\lambda| p(x),\ \forall x \in V,\ \forall \lambda \in \F$ (positive homogeneity);
        \item $p(x + y)\leq p(x) + q(x),\ \forall x, y \in \F$ (triangle inequality).
    \end{enumerate}
    A seminorm is therefore almost a norm, except that it's not necessarily positive definite.
\end{definition}

Let $V$ be a $\F$-vector space and $\mathcal{P}$ a family of seminorms in $V$.
Let $\mathcal{T}$ be the topology in $V$ with the following subbasis:
$$U(x_0, p, \varepsilon) = \{x \in V\ |\ p(x - x_0) < \varepsilon\};\ x_0 \in V,\ p \in \mathcal{P},\ \varepsilon > 0.$$
Basis of $\mathcal{T}$ are finite intersections of such sets.
The set $U \subseteq V$ is open iff for every $x_0 \in U$ there exist seminorms $p_1, \dots, p_n \in \mathcal{P}$
and $\varepsilon_1, \dots, \varepsilon_n > 0$ such that 
$$U \supset \bigcap_{j = 1} ^n U(x_0, p_j, \varepsilon_j).$$
The space $(V, \mathcal{T})$ is then a TVS. If $\mathcal{P}$ is a singleton 
and its element is a norm, then $(V, \mathcal{T})$ is a normed space.

\begin{definition}
    A TVS $X$ is a \emph{locally-convex space} (LCS) if its topology is generated by 
    a family of seminorms $\mathcal{P}$ satisfying 
    $$\bigcap_{p \in \mathcal{P}} \{x \in X\ |\ p(x) = 0\} = \{0\}.$$
\end{definition}

Equivalently, for every $x \in X \setminus \{0\}$ there exists a seminorm $p \in \mathcal{P}$ such that $p(x) \neq 0$.

\begin{corollary}
  Let $X$ be a space with a topology generated by a family of seminorms $\mathcal{P}$. Then $X$ is a LCS iff it is Hausdorff.
\end{corollary}

\begin{myproof}
    Start with $(\Rightarrow)$. Let $x, y \in X$ be two distinct points. There exists a seminorm $p \in \mathcal{P}$
    such that $p(x - y) = b \neq 0$. Define the sets 
    $$V = U\left(x, p, \frac{b}{2}\right),\quad W = U\left(y, p, \frac{b}{2}\right).$$
    By the triangle inequality property of a seminorm, $V$ and $W$ separate the points $x, y$.
    Now the converse $(\Leftarrow)$. Choose a point $X \ni x \neq 0$. Then there exist open sets $0 \in V, x \in W$
    that separate $0$ from $x$. There exists an open basis set $\bigcap_{j = 1} ^n U(0, p_j, \varepsilon_j) \subseteq V$,
    so $x \notin U(0, p_j, \varepsilon_j)$ for some index $j$. Hence, $p_j (x - 0) = p_j(x) \geq \varepsilon > 0$.
\end{myproof}

LCS generally aren't first-countable, so we need to go beyond the usual sequences to describe the topology.

\begin{definition}
    Partially ordered set $(I, \leq)$ is \emph{upwards-directed} if 
    $$\forall i', i'' \in I:\ \exists i \in I:\ i \geq i', i \geq i''.$$
\end{definition}

\begin{example}\label{ex:1.1}
    \begin{enumerate}
        \item Every linearly ordered set is upwards-directed.
        \item Let $(X, \mathcal{T})$ be a topological space and $x_0 \in X$.
        Define a family of sets 
        $$\mathcal{U} = \{U^{\textrm{open}} \subseteq X\ |\ x_0 \in U\}$$
        and a relation $U \geq V \Leftrightarrow U \subseteq V$. Then $(\mathcal{U}, \leq)$ is an upwards-directed set.
        \item Let $S$ be a set and $\mathcal{F}$ a family of all finite subsets of $S$. Define $F_1 \geq F_2$ in $\mathcal{F}$
        if $F_1 \supseteq F_2$. Then $(\mathcal{F}, \leq)$ is again an upwards-directed set.
    \end{enumerate}
\end{example}

\begin{definition}
    A \emph{generalized sequence (net)} is $((I, \leq), x)$, where $(I, \leq)$
    is upwards-directed and $x: I \to X$ is a function. We usually write $(x_i)_{i \in I}$ or $(x(i))_{i \in I}$.
\end{definition}

\begin{example}\label{ex:1.2}
    \begin{enumerate}
        \item Every sequence is a net.
        \item Let $(X, \mathcal{T})$ be a topological space, $x_0 \in X$ and $\mathcal{U}$ a collection of all open sets which contain $x_0$ (see example \ref{ex:1.1}).
        For each $U \in \mathcal{U}$ pick a $x_U \in U$. Then $(x_U)_{U \in \mathcal{U}}$ is a net.
    \end{enumerate}
\end{example}

\begin{definition}
    Let $X$ be a topological space. A net $(x_i)_{i \in I}$ \emph{converges} to an $x \in X$
    if
    $$\forall U^{\textrm{open}} \subseteq X,\ x \in U:\ \exists i_0 \in I:\ \forall i \geq i_0:\ x_i \in U.$$
    We write $\lim_{i \in I} x_i = x$, or alternatively, $x_i \xrightarrow[i \in I]{} x$.
    A point $x \in X$ is called a \emph{cluster point} of a net $(x_i)_{i \in I}$ if 
    $$\forall U^{\textrm{open}} \subseteq X,\ x \in U:\ \forall i_0 \in I:\ \exists i \geq i_0:\ x_i \in U.$$
\end{definition}

\begin{example}
        Take the net $(x_U)_{U \in \mathcal{U}}$ from example \ref{ex:1.2}. It follows from the definition
        that $x_U \xrightarrow[U \in \mathcal{U}]{} x_0$.
\end{example}

\begin{proposition}\label{prop:1.1}
    \begin{enumerate}
        \item Let $X$ be a topological space and $A \subseteq X$. Then $x \in \overline{A}$ iff there exists a net $(a_i)_{i \in I}$ in $A$ such that $a_i \to x$.
        \item Let $X, Y$ be topological spaces and $f: X \to Y$. Then $f$ is continuous at $x_0 \in X$ iff $f(x_i) \to f(x_0)$ for every net $(x_i)_{i \in I}$ that converges to $x_0$.
    \end{enumerate}
\end{proposition}

\begin{myproof}
  \begin{enumerate}
    \item We begin with the implication to the left $(\Leftarrow)$.
    Take any $U^{\textrm{open}} \subseteq X$ such that $x \in U$. Since $a_i \to x$, there exists an index $i_0 \in I$,
    such that for every $i \geq i_0$ we have $a_i \in U$. Hence $a_i \in A \cap U \neq \emptyset$ and $x \in \overline{A}$.
    The converse $(\Rightarrow)$ is similar. Define $\mathcal{U} = \{U^{\textrm{open}} \subseteq X\ |\ x \in U\}$.
    Since $x \in \overline{A}$, for each $U \in \mathcal{U}$, we have $A \cap U \neq \emptyset$.
    Pick $a_U \in A \cap U$. Then the net $(a_U)_{U \in \mathcal{U}}$ in $A$ converges to $x$.
    \item Start with the implication $(\Rightarrow)$. Let $f$ be a continuous function and let $(x_i)_{i \in I}$ converge to $x_0$.
    Let $f(x_0) \in U^{\textrm{open}} \subseteq Y$. Then $x_0 \in f^{-1} (U) ^{\textrm{open}} \subseteq X$, which means there exists an $i_0 \in \N$ 
    such that for every $i \geq i_0$, $x_i \in f^{-1} (U)$. But that implies that for every $i \geq i_0$, $f(x_i) \in U$, which is what we wanted.
    Now we prove the converse $(\Leftarrow)$. Let's say that for every net $(x_i)_{i \in I}$ that converges to $x_0$, we have $f(x_i) \xrightarrow[i \in I]{} f(x_0)$.
    So for every set $A \subseteq X$, we have $f(\overline{A}) \subseteq \overline{f(A)}$ (using the first item),
    which proves that $f$ is continuous. \qedhere
  \end{enumerate}
\end{myproof}

\begin{proposition}
    \begin{itemize}
        \item[(a)] A net $(x_i)_{i \in I}$ in a LCS converges to $x_0$ iff a net $(p(x_i - x_0))_{i \in I}$ converges to $0$ for all $p \in \mathcal{P}$.
        \item[(b)] The topology in a LCS $X$ is the coarsest (smallest) topology in which all the maps 
        $x \mapsto p(x - x_0)$ are continuous for every $x_0 \in X$ and $p \in \mathcal{P}$.
    \end{itemize}
\end{proposition}

\begin{myproof}
    \begin{itemize}
      \item[(a)] Start with the implication $(\Rightarrow)$. Take any $p \in \mathcal{P}$.
      If we take $U = U(x_0, p, \varepsilon)$ in the definition of a limit of a net, we get 
      $$\forall \varepsilon > 0:\ \exists i_0 \in I:\ \forall i \geq i_0:\ p(x_i - x_0) \in (-\varepsilon, \varepsilon).$$
      This proves our claim. Now for the opposite direction $(\Leftarrow)$. For every $p \in \mathcal{P}$ and 
      $\varepsilon_p > 0$ there exists an $i_p$ such that for every $i \geq i_p$, $x_i \in U(x_0, p, \varepsilon_p)$.
      Now let $U$ be an arbitrary basis set that includes the point $x_0$. That means $U$ is the finite intersection of the sets 
      $U(x_0, p, \varepsilon_p)$. Now let $i_0$ be greater than all indices $i_p$. By our assumption, 
      for every $i \geq i_0$ we have $x_i \in U$.
      \item[(b)] Pick any point $x_0 \in X$ and a seminorm $p \in \mathcal{P}$. Denote $$f_{x_0, p}: X \to \R,\quad f_{x_0, p}(x) = p(x - x_0).$$
      We essentially have to prove that the sets
      $$f_{x_0, p}^{-1} (V),\ V^{\textrm{open}} \subseteq \R,\ x_0 \in X,\ p \in \mathcal{P}$$
      generate a subbasis for the seminorm topology of a LCS space.
      Since $f_{p, x_0}$ are continuous functions (by the first item and Proposition \ref{prop:1.1}),
      these are all open sets in the seminorm topology. But on the other hand, all subbasis sets $U(x_0, p, \varepsilon)$
      of the seminorm topology are of this type, so the above subbasis generates the seminorm topology, thus concluding our proof. \qedhere
    \end{itemize}
\end{myproof}

\begin{example}\label{ex:1.3}
    Let $X$ be a topological space.
        For every $K^\textrm{compact} \subseteq X$ we define a seminorm 
        $$p_K: C (X) \to \R,\quad f \mapsto \sup_{x \in K} |f(x)|.$$
        We endow $C (X)$ with the topology induced by the family of seminorms $\{p_K\ |\ K^{\textrm{compact}} \subseteq X\}$. 
        It's trivial to see that $C(X)$ is then a LCS. Moreover, we notice that the induced seminorm topology
        coincides with the topology of compact convergence on $X$. In the future, we will require $X$ to be locally compact Hausdorff (this implies complete regularity)
        so that $C(X)$ has nice properties.
        There are examples of not completely regular spaces $X$ such that the only elements of $C(X)$ are constant maps.
\end{example}

\begin{example}
        Let $D^{\textrm{open}} \subseteq \C$ and let $\mathcal{H} (D)$ be the set of all holomorphic functions on $D$.
        As in the example \ref{ex:1.3}, we define $\mathcal{P} = \{p_K\ |\ K^{\textrm{compact}} \subseteq D\}$.
        This endows $\mathcal{H}(D)$ with a topology and makes $\mathcal{H}(D)$ into a LCS.
        Convergence in this topology concides with the uniform convergence on compacts in $D$.            
\end{example} 

\subsection{Weak topology}
        
    Let $X$ be a normed space and let $X^*$ be its dual. For every $f \in X^*$ we define a seminorm 
        $$p_f: X \to \R,\quad x \mapsto |f(x)|.$$
        We claim that $\mathcal{P} = \{p_f\ |\ f \in X^*\}$ is a family of seminorms that induces a topology on $X$ which makes $X$ a LCS.  
        Indeed, for any $x \in X \setminus \{0\}$ define a nonzero linear functional 
        $$f: \linspan (x) \to \F,\quad f(\lambda x) = \lambda$$
         and extend it to $F: X \to \F$ using Hahn--Banach.
        Then $p_F (x) \neq 0$.
        The induced topology is the \emph{weak topology} on $X$. We denote it as $\sigma(X, X^*)$. 
        
        \begin{proposition}
          A net $(x_i)_{i \in I}$ converges to $x_0 \in X$ with respect to the weak topology iff $f(x_i) \to f(x_0),\ \forall f \in X^*.$          
        \end{proposition}
        
\begin{remark}
        We use the notation $x_i \xrightarrow{w} x_0$. Furthermore, a closure of a set $A \subseteq X$ in the weak topology
        will be denoted by $\overline{A}^w$.  
\end{remark}


\begin{remark}
  The closure of a set $A \subseteq X$ in the weak topology will be denoted as $\overline{A}^w$.
\end{remark}

\begin{example}
  Let $X = \R^n$. Then $X^* = \R^n$ and every linear functional $f$ is of the form 
  $f(x) = \langle x, y \rangle$ for some $y \in X$ (Riesz' representation theorem). The subbasis sets are
  $$U(0, p_y, \varepsilon) = \{x \in \R^n\ |\ |\langle x, y\rangle| < \varepsilon\}.$$
  Weak topology in this case coincides with Euclidean topology.
\end{example} 

    Let $X$ again be a normed space. To $x \in X$ we assign the seminorm $$p_x: X^* \to \R,\quad f \mapsto |f(x)|.$$
    The family $\{p_x\ |\ x \in X\}$ defines a topology in $X^*$ in which $X^*$ becomes a LCS.
    This topology is called the \emph{weak-$*$ topology} and is denoted by $\sigma (X^*, X)$.

\begin{proposition}
    A net $(f_i)_{i \in I}$ converges to $f \in X^*$ with respect to the weak-$*$ topology
    iff $f_i (x)- f (x) \to 0,\forall x \in X$.  
\end{proposition}

\begin{remark}
    We use the notation $f_i \xrightarrow{w^*} f$. Furthermore, a closure of a set $A \subseteq X^*$ in the weak-$*$ topology
    will be denoted by $\overline{A}^{w^*}$.  
\end{remark}

    We can compare weak-$*$ topology on $X^*$ with its weak topology.
    As a consequence of Hahn-Banach, we have for every $x \in X$ $$\| x\| = \sup\{|f(x)||\ f \in X^*,\ \| f\| \leq 1\},$$
    which implies that the map $$\iota: X \hookrightarrow X^{**},\quad x \mapsto (f \mapsto f(x))$$
    is an isometry and therefore injective. This means that every seminorm in the weak-$*$ topology is also 
    a seminorm in a weak topology on $X^*$, so the weak topology is finer (stronger) than the weak-$*$ topology on $X^*$.

\begin{remark}
    Weak and weak-$*$ topology can be defined even if $X$ is merely a LCS.
    In that case, $X^*$ is of course defined as the space of continuous linear functionals on $X$.
\end{remark}

\subsection{Banach-Alaoglu theorem}

\begin{theorem}[Banach-Alaoglu]
    Let $X$ be a normed space. Then the closed unit ball in $X^*$ (denoted by $(X^*)_1$) is compact in the weak-$*$ topology in $X^*$.
\end{theorem}

\begin{myproof}
    To $x \in X$ we assign $D_x = \{z \in \F\ |\ |z| \leq \| x\|\}$
    and endow $D_x$ with the Euclidean topology. Then $D_x$ is clearly compact.
    The set $P = \prod_{x \in X} D_x$ is compact in the product topology (Tychonoff theorem).
    Now we construct a map 
    $$\Phi: (X^*)_1 \to P,\quad f \mapsto (f(x))_{x \in X} \in P.$$
    Clearly, $\Phi$ is well-defined and injective. We start by proving that $\Phi$ is continuous.
    Let $(f_i)_{i \in I}$ be a net in $(X^*)_1$ that converges to $f \in X^*$
    in the weak-$*$ topology. Then $f_i(x) \to f(x)$ for each $x \in X$.
    By the definition of the product topology in $P$, this means that $\Phi(f_i) \mapsto \Phi(f)$ in $P$.
    Hence $\Phi$ is continuous. Since $\Phi$ is injective, it induces an inverse map 
    $$\Phi^{-1}: \im (\Phi) \to (X^*)_1$$
    that is also continuous (we read the previous argument backwards).

    Finally, we prove that $\im(\Phi)$ is closed in $P$. Suppose that $(\Phi(f_i))_{i \in I}$ converges to 
    $p = (p_x)_{x \in X} \in P$. By definition of the product topology, this means that $f_i (x) \to p_x$
    for all $x \in X$. Define $$f: X \to \F,\quad x \mapsto p_x.$$
    Then $f$ is linear and $f \in (X^*)_1$. Thus $p = \Phi(f) \in \im (\Phi)$.
    This in turn implies that $(\im \Phi)^{\textrm{closed}} \subseteq P^{\textrm{compact}}$.
    But we know that $(X^*)_1 \approx \im (\Phi)$, which implies that $(X^*)_1$ is also compact.
\end{myproof}

\begin{corollary}
    Every Banach space $X$ is isometrically isomorphic to a closed subspace 
    of $C (K)$ for some compact $T_2$ space $K$.
\end{corollary}

\begin{myproof}
    Denote $K = (X^*)_1$ endowed with the weak-$*$ topology. By the Banach-Alaoglu theorem, $K$ is compact and $T_2$.
    We now define the map $$\Delta: X \to C(K),\quad x \mapsto (f \mapsto f(x)).$$
    First, we prove that $\Delta$ is isometric. By Hahn-Banach, for every $x \in X \setminus\{0\}$ there exists an $f \in X^*$
    such that $\|f\| = 1$ and $f(x) = \|x\|$. Then we have
    \begin{equation*}
        \| \Delta (x)\|_{\infty} = \sup_{g \in K} |g(x)| = \|x\|.
    \end{equation*} 
    Since $\Delta$ is an isometry, its image is complete and thus closed in $C(K)$.
    Obviously $\Delta$ is a linear map, so we are done.
\end{myproof}

\subsection{Minkowski gauge}

\marginpar{\color{blue} \small ADD MOTIVATION}

\begin{definition}
  Let $X$ be a $\F$-vector space. A set $A \subseteq X$ is 
  \begin{itemize}
    \item balanced if:
    $$\forall x \in A:\ \forall \alpha \in \F,\ |\alpha| \leq 1:\ \alpha x \in A.$$
    \item absorbing if:
    $$\forall x \in X:\ \exists \varepsilon > 0:\ \forall t \in (0, \varepsilon):\ tx \in A.$$
    \item absorbing in $a \in A$ if $A - a = \{x - a\ |\ x \in A\}$ is absorbing.
  \end{itemize}
\end{definition}

\begin{example}
  Let $X$ be a vector space and $p$ a seminorm in $X$. Then 
  $$V = \{x \in X\ |\ p(x) < 1\}$$
  is convex, balanced, absorbing in each of its points.
\end{example}

\begin{theorem}
  Let $X$ be a vector space and $V \subseteq X$ convex, balanced and absorbing in each of its points.
  Then there exists a unique seminorm $p$ on $X$ such that 
  $$V = \{x \in X\ |\ p(x) < 1\}.$$
\end{theorem}

\begin{myproof}
  To $V$ we associate the Minkowski gauge:
  $$p(x) = \inf \{t \geq 0\ |\ x \in t \cdot V\},$$
  where $t \cdot V = \{t \cdot v\ |\ v \in V\}$.
  First we prove that $p$ is well defined.
  Since $V$ is absorbing, we have $X = \bigcup_{n \in \N} n \cdot V$,
  so for every $x \in X$ the set $\{t \geq 0\ |\ x \in t \cdot V\}$
  is nonempty. It's also clear to see that $p(0) = 0$.
  Next we check for homogeneity. Suppose $\alpha \neq 0$.
  Then 
  \begin{align*}
    p(\alpha x) &= \inf \{t \geq 0\ |\ \alpha x \in t\cdot V\}\\
    &= \inf \left\lbrace t \geq 0\ |\ x \in \frac{t}{\alpha}\cdot V \right\rbrace\\
    &= \inf \left\lbrace t \geq 0\ |\ x \in \frac{t}{|\alpha|}\cdot V \right\rbrace\\
    &= \inf |\alpha| \left\lbrace \frac{t}{|\alpha|} \geq 0\ |\ x \in \frac{t}{|\alpha|}\cdot V \right\rbrace\\
    &= |\alpha| p(x).
  \end{align*}
  Now we do the same for triangle inequality:
  let $\alpha, \beta \geq 0$ so that $\alpha + \beta > 0$.
  Let $a, b \in V$. Then 
  $$\alpha a + \beta b = (\alpha + \beta) \left(\frac{\alpha}{\alpha + \beta} a + \frac{\beta}{\alpha + \beta} b\right) \in (\alpha + \beta) \cdot V.$$
  This means that $\alpha \cdot V + \beta \cdot V \subseteq (\alpha + \beta) \cdot V$.
  Now let $x, y \in X$ and $p(x) = \alpha, p(y) = \beta$.
  Take $\delta > 0$. Then $x \in (\alpha + \delta) \cdot V, y \in (\beta + \delta) \cdot V$.
  Hence $$x + y \in (\alpha + \delta) \cdot V + (\beta + \delta) \cdot V \subseteq (\alpha + \beta + 2 \delta) \cdot V,$$
  and by definition, $p(x + y) \leq \alpha + \beta + 2 \delta$.
  Since $\delta > 0$ was arbitrary, we have $p(x + y) \leq \alpha + \beta = p(x) + p(y)$.
  Now that we have proved that $p$ is a seminorm, we can show that 
  $$V = \{x \in X\ |\ p(x) < 1\}.$$
  The inclusion $(\supseteq)$ is easy: if $p(x) < 1$, then $x \in (p(x) + \varepsilon) \cdot V$
  for all $\varepsilon > 0$. By choosing $\varepsilon = 1 - p(x) > 0$, we get $x \in V$.
  Now we prove the other inclusion $(\subseteq)$. Let $x \in V$.
  Since $V$ is absorbing in $x$, there exists an $\varepsilon > 0$ such that 
  $y = x + tx \in V$ for all $t \in (0, \varepsilon)$. This means that $x = \frac{1}{t + 1} y$, where $y \in V$.
  This implies that 
  $$p(x) = \frac{1}{t + 1} p(y) \leq \frac{1}{1 + t} \leq 1,$$
  which proves the equality. Lastly, we prove the $p$ is unique.
  Suppose there is some other seminorm $q$ such that 
  $$\{x \in X\ |\ p(x) < 1\} = \{x \in X\ |\ q(x) < 1\}.$$
  Suppose $p \neq q$. W.l.o.g.~there exists an $x \neq 0$ such that $p(x) > q(x)$.
  By homogeneity, we can assume that $p(x) = 1 > q(x)$, contradicting our assumption.
\end{myproof}

\begin{remark}
  If $X$ is a TVS and $V$ is an open subset, then $V$ is absorbing at each of its points.
\end{remark}

\begin{corollary}
  Let $X$ be a TVS and $\mathcal{U}$ a collection of all open convex balanced subsets of $X$.
  Then $X$ is locally convex iff $\mathcal{U}$ is a basis for the neighborhood system at $0$.
\end{corollary}

\subsection{Applications of Hahn-Banach}

Recall: if $X$ is a $\R$-vector space then $p: X \to \R$
is a sublinear functional if $$p(x + y) \leq p(x) + p(y),\ \forall x, y \in X$$ and $$p(\alpha x) = \alpha x,\ \forall x \in X,\ \alpha > 0.$$

\begin{theorem}[Hahn-Banach theorem]
  \begin{itemize}
    \item[$\R$:] Suppose $X$ is a $\R$-vector space and $p: X \to \R$ is a sublinear functional.
    Given a linear functional $f$ on $Y \leq X$ such that $f(y) \leq p(y)$ for every $y \in Y$,
    $f$ extends to a linear functional $F: X \to \R$ such that $F(x) \leq p(x)$ for every $x \in X$.
    \item[$\C$:] Suppose $X$ is a $\C$-vector space and $p: X \to \R$ is a seminorm.
    Given a linear functional $f$ on $Y \leq X$ such that $|f(y)| \leq p(y)$ for every $y \in Y$,
    $f$ extends to a linear functional $F: X \to \R$ such that $|F(x)| \leq p(x)$ for every $x \in X$.
  \end{itemize}
\end{theorem}

\begin{corollary}[Hahn-Banach extension theorem]
  Let $X$ be a normed space, $f \in X^*$ and $Y \leq X$.
  Then there exists an $F \in X^*$ such that $F\big|_Y = f$ and $\|F\| = \| f\|$.
\end{corollary}

\begin{corollary}[Hahn-Banach separation theorem]
  Suppose $X$ is a LCS and $A, B \subseteq X$ are disjoint closed convex sets.
  If $B$ is compact then there exists an $f \in X^*$ that seperates $A$ from $B$:
  $$\exists \alpha, \beta \in \R:\ \forall a, b \in B:\ \real f(a) \leq \alpha < \beta \leq \real f(b).$$
\end{corollary}

\begin{theorem}\label{thm:1.1}
  Let $X$ be a LCS and $A \subseteq X$ convex. Then $\overline{A} = \overline{A}^w$.
\end{theorem}

\begin{myproof}
  Since the weak topology is weaker than the original topology, we have $\overline{A} \subseteq \overline{A}^w$.
  Let $x \notin \overline{A}$.
  We now separate $\overline{A}$ and the compact set $\{x\}$: there exists $f \in X^*$
  so that there exist $\alpha, \beta \in \R$ and we have
  $$\real f(a) \leq \alpha < \beta \leq \real f(x)$$
  for all $a \in \overline{A}$.
  This means that 
  $$\overline{A} \subseteq \{y \in X\ |\ \real f(y) \leq \alpha\} = (\real f)^{-1} (-\infty, \alpha] = C.$$
  Since $C$ is closed in the weak topology, it follows from $A \subseteq C$ that $\overline{A}^w \subseteq \overline{C}^w = C.$ 
  Since $x \notin C$, we have $x \notin \overline{A}^w$.
\end{myproof}

\begin{corollary}
  A convex set in a LCS is closed iff it is weakly closed.
\end{corollary}

\begin{proposition}\label{prop:1.2}
  Let $X$ be a TVS and $f: X \to \F$ a linear functional. The following are equivalent:
  \begin{enumerate}
    \item $f$ is continuous;
    \item $f$ is continuous in $0$;
    \item $f$ is continuous in some point;
    \item $\ker f$ is closed;
    \item $x \mapsto |f(x)|$ is a seminorm.
  \end{enumerate}
  If $X$ is a LCS, then these are also equivalent to 
  \begin{enumerate}
    \setcounter{enumi}{5}
    \item $\exists \alpha_1, \dots, \alpha_n \in \R_{> 0}$ and $\exists p_1, \dots, p_n \in \mathcal{P}$ such that 
    $$|f(x)| \leq \sum_{k = 1} ^n \alpha_k p_k (x),\ \forall x \in X.$$
  \end{enumerate}
\end{proposition}

\begin{myproof}
  Equivalence of the first five statements is routine.
  Assume that $X$ is a LCS. We prove the equivalence of (2) and (6).
  We start with $(6) \Rightarrow (2)$. Let $(x_i)_{i \in I}$ be a net in $X$ that converges to $0$.
  Then we have 
  $$0 \leq |f(x_i)| \leq \sum_{k = 1} ^n \alpha_k p_k(x_i) \xrightarrow[i \in I]{} 0.$$
  This implies that $f(x_i) \xrightarrow[i \in I]{} 0$, proving the implication.
  Now the opposite: $(2) \Rightarrow (6)$.
  We know that $f^{-1} ( B_1 ^\circ (0)) = \{x \in X\ |\ |f(x)| < 1\}$
  is an open neighborhood of $0$ in $X$. Then there exist $p_1, \dots, p_r \in \mathcal{P}$
  and an $\varepsilon > 0$ such that 
  $$0 \in \bigcap_{i = 1} ^r U(0, p_i, \varepsilon) \subseteq f^{-1} (B_1 ^\circ (0)).$$
  If $p_i(x) < \varepsilon$ for all $i \leq r$, then $|f(x)| < 1.$ Pick any $\delta > 0$.
  Then $$p_i \left(x \cdot \frac{\varepsilon}{\sum p_i (x) + \delta}\right) = \frac{\varepsilon}{\delta + \sum p_i (x)} \cdot p_i (x) < \varepsilon,$$
  which implies 
  $$\left| f\left(x \cdot \frac{\varepsilon}{\sum p_i (x) + \delta}\right) \right| < 1.$$
  From this we get 
  $|f(x)| < \frac{1}{\varepsilon} \left(\sum p_i (x) + \delta\right)$.
  Since $\delta > 0$ was arbitrary, we get
  \begin{equation*}
    |f(x)| \leq \sum_{i = 1} ^r \frac{1}{\varepsilon} p_i(x). \qedhere 
  \end{equation*}
\end{myproof}

Recall the following theorem from measure theory.

\begin{theorem}[Riesz-Markoff theorem]
  Let $X$ be a compact $T_2$ space, $\Phi \in C(X)^*$.
  Then there exists a regular Borel measure $\mu$ such that 
  $$\Phi (f) = \int_X f\, d\mu,\ \forall f \in C(X).$$
  Further, $\| \Phi\| = \|\mu\| = |\mu| (X)$.
\end{theorem}

\begin{remark}
  The above also works if $X$ is locally compact and $\Phi \in C_0(X)^*$.
\end{remark}

As a corollary, we get the following proposition.

\begin{proposition}
  Let $X$ be completely regular. Endow $C(X)$ with a topology induced by its seminorms.
  If $L \in C(X)^{*}$ then there exists a compact $K \subseteq X$ and a regular Borel measure on $K$ such that 
  $$L(f) = \int_{K} f\, d\mu,\ \forall f \in C(X).$$
  Conversely, every such pair $(K, \mu)$ defines $L \in C(K)^*$ with the above equation.
\end{proposition}

\begin{myproof}
  Begin with the implication $(\Leftarrow)$. Given $(K, \mu)$, we just need to prove that the induced 
  functional $L$ is continuous on $X$. We have 
  $$|L(f)| = \left| \int_K f\, d\mu \right| \leq \|\mu\| \sup_K |f| = \|\mu\| p_K (f)$$
  and $L$ is continuous. Now the converse $(\Rightarrow)$.
  Let $L \in C(X)$. By the previous proposition, there exist compact sets 
  $K_1, \dots, K_p \subseteq X$ and $\alpha_1, \dots, \alpha_p > 0$ such that 
  $$|L(f)| \leq \sum_{j = 1} ^p \alpha_j p_{K_j} (f).$$
  Let $K = \bigcup_{j = 1} ^p K_j$ and $\alpha = \max \{\alpha_1, \dots, \alpha_p\}$.
  Then $\|f\| \leq \alpha p_K (f)$ for all $f \in C(X)$.
  Observe that if $f \in C(X)$ and $f \big|_K = 0$, then $L(f) = 0$.
  We now define a map $F: C(K) \to \F$. 
  Since $X$ is completely regular, we have a Tietze-like extension theorem: for any compact $K \subseteq X$ and a continuous function $g \in C(K)$,
  there exists an extension $\widetilde{g} \in C(X)$.
  Define $F(g) := L(\widetilde{g})$.
  First we need to check that $F$ is well defined. Suppose we have two extensions $\widetilde{g}$ and $\widetilde{\widetilde{g}}$
  of $g \in C(K)$. Since $\widetilde{g} - \widetilde{\widetilde{g}}$ is evidently zero on $K$, we have 
  $$L(\widetilde{g}) - L(\widetilde{\widetilde{g}}) = L(\widetilde{g} - \widetilde{\widetilde{g}}) = 0$$
  and $F$ really is well defined. It is also clearly linear, so we just need to check continuity:
  $$|F(g)| = |L(\widetilde{g})| \leq \alpha \cdot p_K (\widetilde{g}) = \alpha \cdot \|g\|_{\infty, K},$$
  therefore $\|F\| \leq \alpha$ and $F$ is continuous. Lastly we apply Riesz-Markoff: there exists a regular Borel measure $\mu$ on $K$
  so that $F(g) = \int_K g\, d\mu.$ If $f \in C(X)$, then $g := f\big|_K \in C(K)$
  and we have 
  \begin{equation*}
    L(f) = F(g) = \int_K g\, d\mu = \int_K f \, d\mu. \qedhere
  \end{equation*}
\end{myproof}

\subsection{Krein-Milman theorem}

\begin{definition}
  Let $X$ be a vector space and $C \subseteq X$ a convex subset.
  \begin{itemize}
    \item[(a)] A nonempty convex subset $F \subseteq C$ is a \emph{face} if for any $x, y \in C$ we have
    $$(\exists t \in (0, 1):\ tx + (1 - t)y \in F) \Rightarrow x, y \in F.$$
    \item[(b)] A point $x \in C$ is a called an \emph{extreme point} if $\{x\} \subseteq C$ is a face.  
  \end{itemize}
  We use the notation $\ext (C)$ for the set of all extreme points of $C$.
\end{definition}

\begin{example}
  If we consider spaces of real sequences, we have
  \begin{itemize}
    \item $\ext ((\ell^\infty)_1) = \{(\pm 1, \pm 1, \dots)\}$;
    \item $\ext ((\ell^1)_1) = \{(0, 0, \dots, \pm 1, \dots)\}$.
  \end{itemize} 
\end{example}

\begin{example}
  We prove that for $c_0$ (the space of complex sequences that converge to $0$) we have $\ext (c_0)_1 = \emptyset$. 
  Indeed, let $x = (x_n)_n \in (c_0)_1$.
  Since $\lim_n x_n = 0$, there exists $N \in \N$ such that $|x_n| < \frac{1}{2}$ for all $n > N$.
  Now define $y, z \in c_0$ by setting $y_n = z_n = x_n$ for $n \leq N$
  and $$y_n = x_n + \frac{1}{2^n},\quad z_n = x_n - \frac{1}{2^n}$$ 
  for $n > N$. Then $y, z \in (c_0)_1$ and $x = \frac{1}{2} (y + z)$, so $x \notin \ext (c_0)_1$.
\end{example}

\begin{example}
  Let us show that $\ext (L^1 [0, 1])_1 = \emptyset$. Take any $f \in (L^1 [0, 1])_1$.
  Then $\int _0 ^1 |f(t)| \, dt = 1$, so there must exist an $x \in [0, 1]$ such that 
  $\int _0 ^x |f(t)| \, dt = 1/2$. Now define 
  $g := 2 \cdot f \cdot \chi_{[0, x]}$ and $h := 2 \cdot f \cdot \chi_{[x, 1]}$. 
  Now we have $g, h \in (L^1 [0, 1])_1$ and $f = \frac{1}{2}g + \frac{1}{2} h$, so $f$ cannot be an extreme point.
\end{example}

\begin{example}
  Finally, let us prove that $\ext (C[0, 1])_1 = \{\pm 1\}$ for real valued functions.
  Take any $f \in (C[0, 1])_1$. Then define functions $g(t) = \min \{2f(t) + 1, 1\}$ and $h(t) = \max \{2 f(t) - 1, -1\}$.
  Clearly $g, h \in (C[0, 1])_1$ and $f = \frac{1}{2} g + \frac{1}{2} h$. If $f$ is an extreme point, then $g = h$, which happens only if $f = \pm 1$.
\end{example}

\begin{definition}
  For a vector space $X$ and $A \subseteq X$, define a \emph{convex hull} $\co A$ as the intersection of all convex sets in $X$ that contain $A$.
  If $X$ is a TVS, then define a \emph{closed convex hull} $\overline{\co} A$ as the intersection of all closed convex sets that contain $A$.
\end{definition}

Convex hull of a set $A$ can be given explicitly:
$$\co A = \left\lbrace \sum_{i = 0}^n \alpha_i x_i\ |\ n \in \N, \alpha_i \geq 0, \sum_{i = 0} ^n \alpha_i = 1, x_i \in A\right\rbrace.$$
If $X$ is a TVS, then $\overline{\co} A = \overline{\co A}$.

\begin{lemma}
  If $C \subseteq X$ is a convex subset of a vector space and $a \in C$, then the following are equivalent.
  \begin{itemize}
    \item[(a)] $a \in \ext C$.
    \item[(b)] If $x_1, x_2 \in C$ and $a = \frac{1}{2} (x_1 + x_2)$, then $x_1 = x_2 = a$.
    \item[(c)] If $x_1, x_2 \in C$, $t \in (0, 1)$ and $a = t x_1 + (1 - t) x_2$, then $x_1 = x_2 = a$.
    \item[(d)] $C \setminus \{a\}$ is a convex set.  
    \item[(e)] If $x_1, \dots, x_n \in C$ and $a \in \co \{x_1, \dots, x_n\}$, then $a = x_k$ for some index $k$.
  \end{itemize}
\end{lemma}

\begin{myproof}
  Items (a) and (c) are equivalent by definition.
  \begin{enumerate}[itemindent=36pt]
    \item[(b) $\Rightarrow$ (c):] Let $a = tx_1 + (1 - t) x_2$. Then $$a = \frac{1}{2} (2t x_1 + (1 - 2t) x_2) + \frac{1}{2} x_2,$$
    so we get $2t x_1 + (1 - 2t) x_2 = x_2$, which gives us $x_1 = x_2$.
    \item[(c) $\Rightarrow$ (d):] Take any $x_1, x_2 \in C \setminus \{a\}$. Since $C$ is convex, $t x_1 + (1 - t) x_2 \in C$.
    Now if $a = t x_1 + (1 - t) x_2 \in \co \{x_1, x_2\}$, then $a = x_1 = x_2$, which contradicts our assumption. So $t x_1 + (1 - t) x_2 \in C \setminus \{a\}$
    and $C \setminus \{a\}$ is convex.
    \item[(d) $\Rightarrow$ (e)] If $x_1, \dots, x_n \in C \setminus \{a\}$, then $\co \{x_1, \dots, x_n\} \subseteq C \setminus \{a\}$ by convexity, contradiction. 
    \item[(e) $\Rightarrow$ (b):] Suppose $a = \frac{1}{2} (x_1 + x_2)$. Then either $x_1 = a$ or $x_2 = a$ by our assumption. W.l.o.g.~assume $x_1 = a$.
    Then $a = \frac{1}{2} (a + x_2)$, which implies $a = x_2$. \qedhere
  \end{enumerate}
\end{myproof}

\begin{lemma}
  Let $X$ be a TVS and $C \subseteq X$ a nonempty compact convex set.
  Then for $\Phi \in X^*$ the set 
  $$F = \{x \in C\ |\ \real \Phi (x) = \min_C \real \Phi\}$$
  is a closed face of $C$.
\end{lemma}

\begin{myproof}
  Since $C$ is compact and $x \mapsto \real \Phi(x)$
  is continuous, it attains its minimum on $C$. Hence $F$ is nonempty.
  Since $F$ is a continuous preimage of a point, it is also closed.
  By the linearity of $\Phi$, $F$ is convex. 
  Now suppose that $t \in (0, 1)$ and $x, y \in C$ are such that 
  $tx + (1 - t)y \in F$. Then
  \begin{align*}
    \min_C \real \Phi &= \real \Phi (tx + (1 - t)y)\\
    &= t \cdot \real \Phi(x) + (1 - t) \real \Phi (y)\\
    &\geq t \cdot \min_C \real \Phi + (1 - t) \min_C \real\\
    &= \min_C \real \Phi. 
  \end{align*}
  Since we have the equality in the second-to-last line, we have 
  $\real \Phi (x) = \min_C \real \Phi$
  and $\real \Phi (y) = \min_C \real \Phi$, meaning that $x, y \in F$.
\end{myproof}

\begin{remark}
  Not all closed convex faces are of this form. \marginpar{\color{blue} \small ADD A PICTURE}
\end{remark}

\begin{theorem}[Krein-Milman]
  Let $X$ be a LCS and $C \subseteq X$ a nonempty compact convex subset.
  Then $C = \overline{\co} (\ext C)$. In particular, $\ext C \neq \emptyset.$
\end{theorem}

\begin{myproof}
  Let $\mathcal{F} = \{\textrm{closed faces in $C$}\}$ be ordered with $\supset$.
  Since $C \in \mathcal{F}$, it is nonempty. The set $\mathcal{F}$ is then partially ordered. Since any 
  increasing chain in $\mathcal{F}$ has the finite intersection property, $\mathcal{F}$ has a nonempty intersection due to $C$ being compact.
  As a result, any increasing chain in $\mathcal{F}$ has an upper bound.
  This tells us that we can apply Zorn's lemma to obtain a maximal element $F_0 \in \mathcal{F}$.
  
  \vspace*{0.5\baselineskip}
  We prove that $F_0 = \{p\}$ for some $p \in X$. Assume for a contradiction that there are distinct $x, y \in F_0$.
  By Hahn-Banach, there exists a $\Phi \in X^*$ such that $\Phi(x) \neq \Phi(y)$.
  W.l.o.g.~we assume that $\real \Phi(x) < \real \Phi(y)$. Define a set 
  $$F_1 = \{z \in F_0\ |\ \real \Phi(z) = \min_{F_0} \real \Phi\}.$$
  Then $F_1 \subsetneq F_0$, since $y \notin F_0$. By the previous lemma,
  $F_1$ is a closed face in $F_0$, so it is a closed face in $C$, contradicting maximality of $F_0$.
  As a result, $F_0 = \{p\}$, which implies that $p \in \ext (C)$ and the set of extreme points of $C$ is non-empty.
  
  \vspace*{0.5\baselineskip}
  Since we have $C \supset \ext C$, we also have $C = \overline{\co} (C) \supseteq \overline{\co} (\ext C)$.
  Suppose $x \in C \setminus \overline{\co} (\ext C)$. By Hahn-Banach, there exists a $\Psi \in X^*$
  such that $\real \Psi(x) < \min _{\overline{\co} (\ext C)} \real \Psi$.
  So the set 
  $$F = \{z \in C\ |\ \real \Psi (z) = \min_C \real \Psi\}$$
  is a closed face in $C$. By the first part of this proof,
  there exists a $z \in \ext F \subseteq \ext C$. Hence 
  $$\min_C \real \Psi = \real \Psi (z) = \min_{\overline{\co} (\ext C)} \real \Psi > \real \Psi(x) \geq \min_C \real \Psi,$$
  which leads to a contradiction. Therefore $\overline{\co} (\ext C) = C$.
\end{myproof}

\begin{example}
  Let $\mathcal{H}$ be a Hilbert space. Then 
  $$\ext (\mathcal{H})_1 = \{v \in \mathcal{H}\ |\ \|v\| = 1\}.$$
  First we prove the inclusion $(\supseteq)$. Suppose that 
  $\| v\| = 1$ and $v = tx + (1 - t)y$, where $t \in (0, 1)$ and $x, y \in (\mathcal{H})_1$.
  We have 
  \begin{align*}
    1 &= \| v\|^2\\
    &= \| tx + (1 - t)y\|^2\\
    &= \langle tx + (1 - t)y, tx + (1 - t)y \rangle\\
    &= t^2 \|x\|^2 + (1 - t)^2 \|y\|^2 + 2t(1 - t)\real \langle x, y\rangle\\
    &\leq t^2 + (1 - t)^2 + 2t(1 - t) = 1.
  \end{align*}
  We get equality in the Cauchy-Schwartz inequality, so $x, y$ are linearly dependent and therefore equal.
  For the reverse inclusion, let $v \in \ext (\mathcal{H})_1$. If $\| v\| < 1$, then 
  $$v = \frac{1}{2} \cdot \frac{v}{\|v\|} + \frac{1}{2} \cdot \left(2{\|v\|} - 1 \right) \frac{v}{\|v\|},$$
  so $v$ cannot be an extreme point of $(\mathcal{H})_1$.
\end{example}

\begin{example}\label{ex:1.4}
  It holds that
  $$\ext (\bh)_1 = \{V \in \bh\ |\ \textrm{$V$ or $V^*$ is an isometry}\}.$$
  Here, we will just prove the inclusion $(\supseteq)$.
  Let $V \in \bh$  be an isometry and suppose $V = tS + (1 - t)T$
  for $t \in (0, 1)$ and $S, T \in (\bh)_1$.
  For $x \in \mathcal{H}$ we have:
  \begin{align*}
    \| x\| &= \| V x\|\\
    &= \| tSx + (1 - t)Tx\|\\
    &\leq t \|Sx\| + (1 - t) \|Tx\|\\
    &\leq t \| S\| \|x\| + (1 - t) \|T\| \|x\|\\
    &\leq t \|x\| + (1 - t)\|x\| = \|x\|. 
  \end{align*}
  Since we have equality, we get $\|S\| = \|T\| = 1$
  and $\|Sx\| = \|Tx\| = \|x\|$. So $S, T$ are isometries.
  For every $x \in \partial (\mathcal{H})_1 = \ext (\mathcal{H})_1$, we have 
  $$Vx = t \cdot Sx + (1 - t) Tx$$ and by the previous example that implies $Tx = Sx = Vx$,
  so we really have $S = T = V$. We use the same argument if $V^*$ is an isometry.
  For now, we lack some tools to prove the reverse inclusion. We will prove the equality in Corollary \ref{cor:4.1}.
\end{example}

\begin{example}
  If $X$ be a Banach space, then $(X^*)_1$ is weak-$*$ compact (by Banach-Alaoglu),
  so Krein-Milman gives us $(X^*)_1 = \overline{\co} (\ext (X^*)_1)$. Hence $(X^*)_1$ has a lot of extreme points.
  As a corollary, $c_0$, $L^1 [0, 1]$ and $C[0, 1]$ are not duals of Banach spaces.
\end{example}

\begin{theorem}[Milman]
  Let $X$ be a LCS, $K \subseteq X$ compact and assume $\overline{\co} (K)$ is compact.
  Then $\ext (\overline{\co} (K)) \subseteq K$.
\end{theorem}

\begin{myproof}
  Assume there exists $x_0 \in \ext (\overline{\co} (K)) \setminus K$.
  Then there exists a basis neighborhood $V$ of $0$ in $X$ such that 
  $(x_0 + \overline{V}) \cap K = \emptyset$, or equivalently, $x_0 \notin K + \overline{V}$.
  If we write 
  $K \subseteq \bigcup_{x \in K} (x + V)$, 
  we get 
  $$K \subseteq \bigcup_{j = 1} ^{n} (x_j + V).$$
  Form $K_j = \overline{\co} (K \cap (x_j + V))$.
  Then $K_j$ is convex and compact since $K_j \subseteq \overline{\co} (K)$.
  We also have $K_j \subseteq \overline{x_j + V} = x_j + \overline{V}$ since $V$ is convex.
  Also, $K \subseteq K_1 \cup \cdots \cup K_n$.
  Next we prove that $\co (K_1 \cup \cdots \cup K_n)$ is compact.
  Define $$\Sigma = \{(t_1, \dots, t_n) \in [0, 1]^n\ |\ \sum_{j = 1} ^n t_j = 1\}$$
  and the function 
  $$f: \Sigma \times K_1 \times \cdots \times K_n \to X,\quad (t, k_1, \dots, k_n) \mapsto \sum_{j = 1} ^k t_j k_j.$$
  Denote $C := \im f$.
  Obviously, $C \subseteq \co (K_1 \cup \cdots \cup K_n)$ and $C$ is a convex compact set.
  Furthermore, $C \supset K_j$ for each $j$, so $C = \co (K_1 \cup \cdots \cup K_n)$.
  From there, we get 
  $$\overline{\co} (K) \subseteq \overline{\co} (K_1 \cup \cdots \cup K_n) = {\co} (K_1 \cup \cdots \cup K_n).$$
  But since $K_j \subseteq \overline{\co} (K)$ for all $j$, we deduce $\overline{\co} (K) = \overline{\co} (K_1 \cup \cdots \cup K_n)$.
  We know that $x_0 \in \overline{\co} (K)$, so $$x_0 = t_1 y_1 + \cdots + t_n y_n$$
  for some $t_i \in [0, 1],\ \sum t_i = 1$ and $y_j \in K_j$.
  But $x_0 \in \ext (\overline{\co})(K)$, so $y_j = x_0$ for some $j$. So we get 
  $x_0 \in K_j \subseteq x_j + \overline{V} \subseteq K + \overline{V}$,
  a contradiction.
\end{myproof}

\begin{remark}
  \begin{enumerate}
    \item In finite dimensions, the convex hull of a compact set is compact.
    In infinite dimensions this fails.
    \item The set $\ext (C)$ is not always closed, even if $C \subseteq \R^3$ is convex and compact. \marginpar{\color{blue} \small ADD A PICTURE}
  \end{enumerate}
\end{remark}