\section{von Neumann algebras}

\subsection{Bicommutant theorem}

\begin{definition}
  A \emph{von Neumann algebra} (on Hilbert space $\mathcal{H}$) is a $*$-subalgebra of $\mathcal{B}(\mathcal{H})$
  that is WOT-closed. Equivalently, it is a $*$-subalgebra of $\bh$ that is SOT-closed.
\end{definition}

\begin{remark}
  To shorten the notation, we will abbreviate ˝von Neumann algebra˝ to vNa.
\end{remark}

If $A \subseteq \bh$, then $W^* (A)$ denotes the vNa generated by $A$, or 
the smallest vNa in $\bh$ that contains $A$.
This is well defined, since
$$W^* (A) = \bigcap \{W \ |\ A \subseteq W,\ \textrm{$W \subseteq \bh$ is vNa} \}.$$

\begin{lemma}
  If $A \subseteq \bh$ is a vNa, then $(A)_1$ is WOT-compact.
\end{lemma}

\begin{myproof}
  By corollary \ref{cor:4.2}, $(\bh)_1$ is compact in $\sigma$-WOT topology.
  By corollary \ref{cor:4.3}, the WOT and $\sigma$-WOT topologies on $(\bh)_1$ are equivalent,
  so $(\bh)_1$ is also compact in WOT topology. Next, we prove that $(A)_1$ is WOT-closed in $(\bh)_1$.
  Suppose that the net $(x_i)_i$ in $(A_1)$ converges to some $x$. Since $A$ is WOT-closed, we must have $x \in A$. Assume that $x \notin (A)_1$,
  so $\| x\| > 1$. Since $\| x\| = \sup_{\alpha, \beta \in (\mathcal{H})_1} |\langle x\alpha, \beta\rangle|$, 
  there must exist some $\alpha, \beta \in (\mathcal{H})_1$ such that $|\langle x\alpha, \beta\rangle| > 1$.
  However, for every $x_i$ we have $|\langle x_i \alpha, \beta\rangle| \leq \| x_i\| \cdot |\alpha| \cdot |\beta| \leq 1$,
  contradicting the fact that $\langle x_i \alpha, \beta\rangle \to \langle x \alpha, \beta\rangle$.
  Therefore, $x \in (A)_1$ and $(A_1)$ is WOT-closed in $(\bh)_1$, so it is compact.
\end{myproof}

\begin{corollary}
  Let $A \subseteq \bh$ vNa. Then $(A)_1$ and $A_{\sa}$ are SOT-closed and WOT-closed.
\end{corollary}

\begin{myproof}
  We already know that the adjoint is continuous in WOT, so $A_{\sa}$ is closed in WOT.
  Since $A_{\sa}$ is convex, it is also SOT-closed.
  The same exact argument applies for $(A)_1$.
\end{myproof}

\begin{definition}
  The \emph{commutant} of a set $B \subseteq \bh$ is 
  $$B' := \{T \in \bh\ |\ \forall S \in B:\ ST = TS\}$$
  and its \emph{bicommutant} is $B'' := (B')'$. 
\end{definition}

\begin{remark}
  By definition, $B'' \supseteq B$.
\end{remark}

\begin{theorem}\label{thm:5.1}
  Suppose $A \subseteq \bh$ is closed under $*$. Then $A'$ is vNa.
\end{theorem}

\begin{myproof}
  Obviously, $A'$ is also a subalgebra of $\bh$ that is closed under $*$. We prove that it is WOT-closed.
  Let $(x_\alpha)_{\alpha}$ be a net in $A'$ that WOT-converges to $x \in \bh$.
  Pick any $a \in A$ and $\varphi, \mu \in \mathcal{H}$. Then 
  \begin{align*}
    \langle [x, a]\varphi, \mu\rangle &= \langle (xa - ax) \varphi, \mu\rangle\\ 
    &= \langle x a \varphi, \mu\rangle - \langle a x\varphi, \mu\rangle\\
    &= \langle x a \varphi, \mu\rangle - \langle x\varphi, a^*\mu\rangle\\
    &= \lim_{\alpha} \langle x_\alpha a \varphi, \mu\rangle - \langle x_\alpha \varphi, a^*\mu\rangle\\
    &= \lim_{\alpha} \langle (x_\alpha a - a x_\alpha) \varphi, \mu\rangle\\
    &= \lim_{\alpha} \langle [x_\alpha, a]\varphi, \mu\rangle = 0,
  \end{align*}
  so $x \in A'$ and we are done.
\end{myproof}

\begin{corollary}
  Every vNa is unital.
\end{corollary}

\begin{example}
  For an infinitely-dimensional Hilbert space $\mathcal{H}$, the set of all compact operators $\mathcal{K}(\mathcal{H})$
  is not a vNa, since it doesn't include the identity (by the Riesz lemma).
  In particular, $\mathcal{K} (\mathcal{H})$ is neither SOT- nor WOT-closed.
\end{example}

\begin{remark}
  As we will see later, the finite-rank projections on a Hilbert space converge strongly to identity.
\end{remark}

\begin{corollary}
  Suppose that $A \subseteq \bh$ is a maximal commutative subalgebra. If $A$ is closed under $*$,
  then it is a vNa. 
\end{corollary}

\begin{myproof}
  Since $A$ is commutative, $A' \supseteq A$. Take $b \in A' \subseteq A$ and consider the subalgebra, generated by $A$ and $b$. This is an abelian algebra, so by maximality we have 
  we have $b \in A$ and $A = A'$. Then by theorem \ref{thm:5.1}, $A$ is a vNa.
\end{myproof}

\begin{lemma}\label{lem:5.1}
  Let $A \subseteq \bh$ be a $*$-subalgebra. For any $\mu \in \mathcal{H}$ and $x \in A''$
  there exists a net $(x_\alpha)_\alpha$ in $A$ such that $\lim_\alpha \| (x_\alpha - x)\mu \| = 0$.
\end{lemma}

\begin{myproof}
  Define $\mathcal{K} := \overline{A \mu} \leq \mathcal{H}$. Let $p: \mathcal{H} \to \mathcal{K}$ be the orthogonal projection onto $\mathcal{K}$.
  By definition, $a \mathcal{K} \subseteq \mathcal{K}$ for any $a \in A$.
  Equivalently, $pap = ap$ for any $a \in A$. Then 
  $$pa = (a^* p) ^* = (p a^* p)^* = pap = ap,$$
  so $p \in A'$. But $x \in A''$, so  
  $$xp = xp^2 = pxp$$
  and $x \mathcal{K} \subseteq \mathcal{K}$. In particular, since $\mu \in \mathcal{K}$, we have $x \mu \in \mathcal{K} = \overline{A\mu}$.
  So there must exist some net in $A \mu$ that converges to $x \mu$.
\end{myproof}

\begin{theorem}[von Neumann's bicommutant theorem]
  Let $A \subseteq \bh$ be a $*$-subalgebra. Then $\overline{A}^{\mathrm{WOT}} = A''$.
\end{theorem}

\begin{myproof}
  By the previous theorem, $A''$ is a vNa. In particular, it is WOT-closed.
  Since $A \subseteq A''$, it suffices to show that $A$ is WOT-dense in $A''$.
  Because $A$ is convex, it is enough to show that $A$ is SOT-dense in $A''$.
  Let $x \in A''$ and $\mu_1,\dots, \mu_n \in \mathcal{H}$. 
  Consider the matrix $*$-algebra $M_n (\bh)$ with the usual matrix involution.
  There exists a canonical $*$-isomorphism $M_n (\bh) \to \mathcal{B} (\mathcal{H}^n)$, which allows us to introduce 
  a (unique) norm on $M_n (\bh)$, making it a $C^*$-algebra.
  Define 
  $$\widetilde{A} = \left\lbrace \begin{bmatrix}
    a & &\\
    & \ddots &\\
    & & a
  \end{bmatrix} \in M_n (\bh)\ |\ a \in A\right\rbrace.$$
  Then $\widetilde{A}' = M_n (A')$.
  Hence we get 
  $$\widetilde{A''} \subseteq M_n (A') ' = \widetilde{A}''.$$
  This implies that $$\begin{bmatrix}
    x & &\\
    & \ddots &\\
    & & x
  \end{bmatrix} \in \widetilde{A''} \subseteq \widetilde{A}''.$$
  Now we apply lemma \ref{lem:5.1} to $\widetilde{A}$ to get a net $(a_i)_i$ in $A$
  such that 
  \begin{equation*}
    \lim_{i} \| (x - a_i) \mu_j \| = 0,\quad \forall j = 1, \dots, n.
  \end{equation*}
  Finally, we have to show that this implies that $x$ is in the SOT-closure of $A$.
  Let $U$ be an open neighborhood around $x$. Then $U$ must contain some finite intersection of 
  subbasis sets that generate the SOT topology. This means that there exists $\varepsilon > 0$ and
  $\mu_1, \dots, \mu_n \in \mathcal{H}$ such that 
  $$\bigcap_{j = 1} ^n \{y \in \bh\ |\ \|(x - y)\mu_j\| < \varepsilon\} \subseteq U.$$
  Now we can conclude that $U \cap A \neq \emptyset$ and $x$ is in the SOT-closure of $A$.
\end{myproof}

\begin{corollary}
  Let $A \subseteq \bh$ be a $*$-subalgebra. Then $A$ is a vNa iff $A = A''$.
\end{corollary}

\begin{remark}
  WOT-closed implies norm-closed. In particular, every vNa is a $C^*$-algebra.
  However, the converse is not always true: $\mathcal{C} ([0, 1])$ is a $C^*$-algebra that is not vNa.
  As we will see, this is because it does not contain nontrivial projections.
\end{remark}

\begin{corollary}[Polar decomposition in vNa]
  Let $A \subseteq \bh$ be a vNa and $x \in A$.
  Suppose that $x = v|x|$ is the polar decomposition of $x$ in $\bh$.
  Then $v \in A$.
\end{corollary}

\begin{myproof}
  We know that 
  $$\ker v = (\im |x|)^\perp = \ker |x| = \ker x.$$
  For $a \in A'$ and $\mu \in \ker x$ we have $a\mu \in \ker x$:
  $$x (a\mu) = a x\mu = 0,$$
  which implies $a \ker |x| \subseteq \ker |x|$.
  We know that $\mathcal{H} = \ker |x| \oplus \overline{\im |x|}$.
  Suppose that $|x| \mu \in \im |x|$. Then 
  \begin{align*}
    [a, v] |x| \mu &= (av - va) |x| \mu = av |x| \mu - va |x| \mu\\
    &= ax\mu - v|x| a \mu = ax\mu - xa\mu \\
    &= [a, x]\mu = 0.
  \end{align*}
  But for $\beta \in \ker |x| = \ker v$, we have
  $$[a, v] \beta = (av - va)\beta = av\beta - va \beta = 0.$$
  Since $av$ and $va$ agree on $\ker |x| \oplus \overline{\im |x|}= \mathcal{H}$, we have $v \in A'' = A$.
\end{myproof}

The next example is of fundamental importance.

\begin{example}[Commutative vNa] \label{ex:5.1}
  Let $(X, \mu)$  be a $\sigma$-finite measure space and 
  $$M: L^\infty (X, \mu) \to \mathcal{B} (L^2 (X,\mu)),\quad g \mapsto M_g,$$
  where we define 
  $$(M_g f) (x) = g (x) f(x).$$
  Then $M$ is an isometric $*$-isomorphism onto its image
  and $M(L^\infty (X, \mu))$ is a maximal commutative vNa in $\mathcal{B} (L^2 (X, \mu))$.
\end{example}

\begin{comment}
  \begin{remark}
    A measurable function $f: X \to \C$ is essentially bounded if there exists a real number $M$ (called an essential bound) such that 
    $$\mu \left(\left\lbrace x \in X\ |\ |f(x)| > M \right\rbrace\right) = 0.$$
    We define $\| f\|_{\infty}$ to be an essential supremum of $f$, which is the infimum of its essential bounds.
    If $\mathcal{L}^\infty (X, \mu)$ are essentially-bounded functions and $\mathcal{N} = \{f\ |\ \| f\|_\infty  = 0\}$.
    Then $\| \cdot \|_{\infty}$ is a norm on the space $L^\infty (X, \mu) = \quot{\mathcal{L}^\infty (X, \mu)}{\mathcal{N}}$.
  \end{remark}  
\end{comment}


\begin{myproof}[Proof of the example]
  Clearly, $M$ is injective, additive and multiplicative.
  First, we prove that $M$ is a $*$-homomorphism. This follows from the next calculation:
  \begin{align*}
    \langle M_{\overline{g}} \mu, \varphi \rangle &= \int_X M_{\overline{g}} \mu \cdot \overline{\varphi}\, d\mu\\
    &= \int_X \overline{g}\mu \overline{\varphi}\\
    &= \int_X \mu \overline{g \varphi}\, d\mu\\
    &= \langle \mu, M_g\varphi\rangle = \langle M_g ^* \mu, \varphi \rangle,
  \end{align*}
  so $M_{\overline{g}} = M_g ^*$. Next, we prove that $M$ is isometric.
  For $g \in L^\infty (X, \mu)$, there exists a sequence $E_n \subseteq X$ such that $0 < \mu(E_n) < \infty$
  and $|g|\big|_{E_n} \geq \|g\|_{\infty} -\frac{1}{n}$ for all $n \in \N$.
  Then 
  $$\| M_g \| \geq \frac{\| M_g 1_{E_n}\|_2}{\| 1_{E_n}\|_2} \geq \|g\|_{\infty} - \frac{1}{n},\quad \forall n \in \N,$$
  which implies $\|M_g\| \geq \|g\|_{\infty}$.
  For the reverse, notice that 
  \begin{align*}
    \| M_g 1_{E_n} \|^2 &= \int_X |g \cdot 1_{E_n}|^2\, d\mu\\
    &= \int_{E_n} |g|^2\, d\mu\\
    &\geq \int_{E_n} (\|g\|_{\infty} - \frac{1}{n})^2\, d\mu \\
    &= (\|g\|_\infty -\frac{1}{n})^2 \cdot \mu(E_n)
  \end{align*}
  and 
  \begin{align*}
    \|M_g\|^2 &= \sup_{\| \mu\|_2 = 1} \| M_g \mu\|^2 _2 = \sup_{\| \mu\|_2 = 1} \int_X |g\mu|^2\, d\mu\\
    &\leq \|g\|_{\infty} ^2 \cdot \sup_{\| \mu\|_2 = 1} \int_X |\mu|^2\, d\mu = \|g\|_{\infty} ^2.
  \end{align*}
  We've just shown that $\|Mg\| = \|g\|_{\infty}$.
  Lastly, we prove that $M(L^\infty (X, \mu))$ is a maximal commutative subalgebra of $\mathcal{B}(L^2 (X, \mu))$.
  Take $T \in \mathcal{B}(L^2 (X, \mu))$ and assume it commutes with all $M_g$'s.
  Now pick a measurable sequence $E_n \subseteq X$ such that $0 < \mu(E_n) < \infty$, $E_n \subseteq E_{n + 1}$ and $X = \bigcup_{n \in \N} E_n$.
  Define $f_n := T(1_{E_n}) \in (X, \mu)$. First we prove that $f_n \in L^\infty (X, \mu)$. If $A$ is measurable and $0 < \mu(A) < \infty$, then 
  \begin{align*}
    \frac{1}{\mu(A)} \int_X |f_n \cdot 1_A|^2\, d\mu &= \frac{1}{\mu(A)} \cdot \|M_{1_A} T(1_{E_n})\|^2\\
    &= \frac{1}{\mu(A)} \cdot \| T(1_{A \cap E_n}) \|^2\\
    &\leq \frac{1}{\mu(A)} \cdot \|T\|^2 \cdot \|1_A \|^2 = \| T\|^2.
  \end{align*}
  If $f \notin L^\infty(X, \mu)$, then for all $M \in \R$ we have 
  $$0 < \mu (\underbrace{\{x \in X\ |\ |f_n (x)| > M\}}_{A_{n, M}}) < \infty,$$
  since $f_n \in L^2(X, \mu)$. By above calculation,
  $$M^2 \leq \frac{1}{\mu(A_{n, M})} \cdot \int_X |f \cdot 1_{A_{n, M}} |^2\, d\mu \leq \|T\|^2,$$
  which is of course a contradiction. This proves that $f_n \in L^\infty (X, \mu)$ and $\| f_n\|_{\infty} \leq \|T \|.$
  For $n \leq m$ we have 
  \begin{align*}
    1_{E_n} \cdot f_m &= 1_{E_n} \cdot T(1_{E_m})\\
    &= M_{1_{E_n}} (T (1_{E_m}))\\
    &= T(M_{1_{E_n}} 1_{E_m})\\
    &= T(1_{E_n} 1_{E_m}) = f_n.
  \end{align*}
  Therefore, $f_m \big|_{E_n} = f_n$. The sequence $(f_n)_n$ converges to a measurable $f: X \to \C$.
  From $\| f_n\|_{\infty} \leq \|T\|$ for all $n \in \N$ we also deduce $\|f\|_{\infty} \leq T$,
  so $f \in L^\infty (X, \mu)$. Lastly, we prove $T = M_f$.
  Note that simple functions $\sum_{j = i} ^r \alpha_j 1_{A_j}$ are $L^2(X, \mu)$-dense.
  Let $A \subseteq X$ be measurable with $\mu(A) < \infty$.
  Then $\| 1_{A \cap E_n} - 1_A\|_2 \xrightarrow{n \to \infty} 0$.
  Hence $$\| (T - M_f) 1_A \|_2 = \lim_{n \to \infty} \| (T - M_f) 1_{A \cap E_n} \|_2 = 0,$$ as we shall prove.
  \begin{align*}
    T(1_{A \cap E_n}) &= T(1_A \cdot 1_{E_n}) = T(M_{1_A} 1_{E_n})\\
    &= M_{1_{A}} (T(1_{E_n})) = M_{1_A} (f_n)\\
    &= 1_A \cdot f_n.
  \end{align*}
  On the other hand,
  $$M_f (1_{A \cap E_n}) = f \cdot 1_{A \cap E_n} = f \cdot 1_{E_n} \cdot 1_A = 1_A \cdot f_n$$
  and we are done.
\end{myproof}


Another possible characterization of vNa's is given by the following.

\begin{theorem}[Sakai]
  Let $A$ be a $C^*$-algebra such that for a Banach space $E$ there exists an isometric isomorphism 
  $A \to E^*$. Then there exists a vNa $B \subseteq \bh$ such that $A \cong B$ as a $C^*$-algebra.
\end{theorem}

For the proof, see the expository article \cite{kadison}. 

\subsection{Kaplansky's density theorem}

\begin{lemma}
  The multiplication $(A, B) \mapsto A \cdot B$ is SOT-continuous on bounded sets.
\end{lemma}

\begin{myproof}
  Let $(A_i)_i$ and $(B_i)_i$ be nets with 
  $\sup \| A_i\|, \sup \|B_i\| < M$ for some $M \in \R$.
  Suppose $A_i \to A$ and $B_i \to B$ in SOT. For any $x$, we get 
  \begin{align*}
    \|ABx - A_i B_i x\| &= \| ABx - A_i Bx + A_i Bx - A_i B_i x\|\\
    &\leq \|AB x - A_i Bx\| + \|A_i Bx - A_i B_ix\|\\
    &\leq \|A(Bx) - A_i (Bx)\| + \|A_i\| \cdot \|Bx - B_i x\|\\
    &\leq \|A(Bx) - A_i (Bx)\| + M \cdot \|Bx - B_i x\|\to 0,
  \end{align*}
  so $A_i B_i \xrightarrow{\textrm{SOT}} AB$.
\end{myproof}

\begin{proposition}
  Let $f \in C (\C)$. Then $x \mapsto f(x)$ is SOT-continuous on each bounded set of normal operators in $\mathcal{B}(\mathcal{H})$.
\end{proposition}

\begin{myproof}
  By Stone--Weierstrass, we can uniformly approximate $f$ by polynomials on a bounded subset $B_R (0) \subseteq \C$.
  By the previous lemma, multiplication is SOT-continuous on this bounded set of normal operators.
  But for a normal operator $A$, we have $\|Ax\| = \|A^*x\|$ for every $x\in \mathcal{H}$, so $*$ is also SOT-continuous on normal operators and we're done.  
\end{myproof}

\begin{theorem}[Cayley transform]
  Mapping $x \mapsto (x - i)(x + i)^{-1}$ is SOT-continuous $\bh_{\sa} \to \mathcal{U} (\mathcal{H})$.
\end{theorem}

\begin{myproof}
  If $x \in \mathcal{B}(\mathcal{H})_{\sa}$, then $\sigma(x) \subseteq \R$ and $(x + i) \in \bh$ is invertible.
  We notice that $z \mapsto \frac{z - i}{z + i}: \R \to \C$ has its range in $\mathbb{T}$,
  so the Cayley transform does in fact map into the unitaries. Now onto the SOT-continuity:
  let $(x_k)_k$ be a net in $\mathcal{B}(\mathcal{H})_{\sa}$ with $x_k \to x$ in SOT.
  By the spectral mapping theorem, $\| (x_k + i)^{-1} \| \leq 1.$ For each $\alpha \in \mathcal{H}$,
  we have
  \begin{align*}
    \|(x - i) (x + i)^{-1} \alpha - (x_k - i)(x_k + i)^{-1} \alpha\| &= \| (x_k + i)^{-1} \left((x_k + i) (x - i) (x + i^{-1}) - (x_k - i)\right) \alpha \|\\
    &= \| (x_k + i)^{-1} \left((x_k + i) (x - i) - (x_k - i)(x + i)\right) (x + i)^{-1} \alpha \|\\
    &= \| (x_k + i)^{-1} 2i (x - x_k) (x + i)^{-1} \alpha\|\\
    &\leq 2 \|(x_k + i)^{-1}\| \|(x - x_k)\underbrace{(x + i)^{-1} \alpha}_{\beta}\|\\
    &\leq 2\| (x - x_k) \beta\| \to 0. \qedhere
  \end{align*}
\end{myproof}

\begin{corollary}
  If $f \in C_0(\R)$, then $x \mapsto f(x)$ is SOT-continuous on $\mathcal{B}(\mathcal{H})_{\sa}$.
\end{corollary}

\begin{myproof}
  Consider the continuous function 
  $$g(t) = \begin{cases}
    f\left(i \frac{1 + t}{1 - t}\right);& t \neq 1\\
    0;& t = 1
  \end{cases}$$
  which maps $\mathbb{T} \to \C$.
  By the previous proposition, $x \mapsto g(x)$ is SOT-continuous on unitaries.
  Letting $U(z) = \frac{z - i}{z + i}$, denote the Cayley transform, we have that 
  $f = g \circ U$ is a composite of two SOT-continuous maps, which is a SOT-continuous map of itself.
\end{myproof}

\begin{theorem}[Kaplansky's density theorem]
  Let $A \subseteq \bh$ be a $*$-subalgebra and $B = \overline{A}^{\mathrm{SOT}}$,
  then \begin{enumerate}
    \item $\overline{A_{\sa}}^{\mathrm{SOT}} = B_{\sa}$;
    \item $\overline{(A)_1}^{\mathrm{SOT}} = (B)_1$.
  \end{enumerate}
\end{theorem}

\begin{myproof}
  W.l.o.g. $A$ is a $C^*$-algebra, so norm-closed.
  \begin{enumerate}
    \item First we prove that $\overline{A_{\sa}}^{\mathrm{SOT}} \subseteq B_{\sa}$.
    Since $\overline{A_{\sa}}^{\mathrm{SOT}} = \overline{A_{\sa}}^{\mathrm{WOT}}$, take $x \in \overline{A_{\sa}}^{\mathrm{SOT}}$
    and a net $(x_k)_k \subseteq A_{\sa}$ that converges to $x$. Since $*$ is WOT continuous, 
    $(x_k ^*)_k = (x_k)_k$ converge to $x^*$, so $x = x^*$.
    Now the converse inclusion: suppose the net $(x_k)_k$ SOT-converges to $x \in B_{\sa}$.
    Then $\frac{x_k + x_k^*}{2} \to x$ in the WOT-topology, which implies 
    $$B_{\sa} \subseteq \overline{A_{\sa}}^{\mathrm{WOT}} = \overline{A_{\sa}}^{\mathrm{SOT}}.$$
    \item Suppose the net $(y_i)_i$ in $A_{\sa}$ SOT-converges to $x \in B_{\sa}$.
    Take $f \in C_0 (\R)$ such that we have $f(t) = t,\ \forall |t| \leq \|x\|$ and $|f(t)| \leq \|x\|,\ \forall t \in \R$. 
    By functional calculus, $\| f(y_k)\| \leq \|x\|$. By the previous corollary, $(f(y_i))_i \xrightarrow{\mathrm{SOT}} f(x) = x$.
    This proves that $(A)_1 \cap A_{\sa}$ is SOT-dense in $(B)_1 \cap B_{\sa}$.
    Pass over to $M_2 (\bh) = \mathcal{B} (\mathcal{H} \oplus \mathcal{H})$.
    Then $M_2 (A)$ is SOT-dense in $M_2 (B)$ by the first part of the proof. For $x \in (B)_1$, we have 
    $$\widetilde{x} = \begin{pmatrix}
      0 & x\\
      x^* & 0
    \end{pmatrix} \in (M_2 (B))_1 \cap (M_2 (B))_{\sa}.$$ 
    That means there exists a net $$\widetilde{x_i} = \begin{pmatrix}
      a_i & b_i\\
      c_i & d_i
    \end{pmatrix} \in (M_2 (A))_1$$ such that $\widetilde{x_i} \to \widetilde{x}$
    and therefore $b_i \in (A)_1$ SOT-converge to $x$. \qedhere
  \end{enumerate}
\end{myproof}

\begin{corollary}
  Let $A \subseteq \bh$ be a $*$-algebra. Then $A$ is a vNa iff $(A)_1$ is SOT-closed.
\end{corollary}

\subsection{Examples of vNa's}

\begin{definition}
  A vNa $M$ is called a \emph{factor} if $Z(M) = M \cap M' = \C \cdot 1$.
\end{definition}

\begin{example}
  Clearly, $\bh$ is a factor. In particular, $M_n (\C)$ is a factor.
\end{example}

Let $\Gamma$ be a group and $\mathcal{H} = \ell^2 (\Gamma)$. Consider the left regular representation
$$\lambda: \Gamma \to \mathcal{B} (\ell^2 (\Gamma)),\quad g \mapsto (\delta_h \mapsto \delta_{gh})$$
and extend it linearly to $\lambda: \C [\Gamma] \to \mathcal{B} (\ell^2 (\Gamma)).$
The group vNa of $\Gamma$ is $VN (\Gamma) := \lambda (\C [\Gamma]) ''$
in $\mathcal{B} (\ell^2 (\Gamma))$. It has a \emph{trace}, which is defined as the linear functional 
$$\tau: VN(\Gamma) \to \C,\quad x \mapsto \langle x \delta_e, \delta_e\rangle.$$
For $g \in \Gamma$, $\tau (\lambda (g)) = 1$ if $g = e$, otherwise zero.
For $g_1, \dots, g_r \in \Gamma$, we have 
$$g_1 \dots g_r = e \Leftrightarrow \tau (\lambda (g_1) \dots \lambda (g_r)) = 1.$$
Since $\tau$ is a positive linear functional and $\tau(1) = 1$, $\tau$ is a state.
For any two elements $g, h \in \Gamma$ we have $gh = e \Leftrightarrow hg = e$, which together with the above line implies 
$$\tau(\lambda(g) \lambda(h)) = \tau (\lambda(h) \lambda (g)).$$
By linearity, $\tau$ has the same cyclic property on $\lambda (\C [\Gamma])$.
But since $\tau$ is, by definition, WOT-continuous and $VN(\Gamma) = (\lambda(\C [\Gamma]))'' = \overline{\lambda(\C [\Gamma])}^{\textrm{WOT}}$,
$\tau$ is cyclic on the entire $VN(\Gamma)$. Now if $|\Gamma| = \infty$,
then $VN(\Gamma) \neq \bh$, since the latter does not have a trace if $\dim \mathcal{H} = \infty$.
If $\Gamma$ is an abelian group, then $VN(\Gamma)$ is commutative. 

\begin{definition}
  Group $\Gamma$ has \emph{icc} (infinite conjugacy classes), if for all $g \in \Gamma \setminus \{e\}$,
  the set $\{f^{-1} g f\ |\ f \in \Gamma\}$ is infinite.
\end{definition}

\begin{example}
  The group 
  $$S_{\infty} = \{\textrm{bijections $\N \to \N$ that only permute finitely many elements}\}$$ has icc.
\end{example}

\begin{example}
  Free groups $\F_n$ for $n > 1$ have icc.
\end{example}

\begin{theorem}
  If $\Gamma$ has icc, then $VN(\Gamma)$ is a factor.
\end{theorem}

\begin{definition}
  $VN(S_\infty) =: R$ is the \emph{hyperfinite II$_1$-factor}.
\end{definition}

Open problem: does $VN (\F_2) \cong VN (\F_3)$ hold?

\subsection{Operations with vNa's}

\subsubsection{Direct sums}

Let $M_i \subseteq \mathcal{B}(\mathcal{H}_i)$ be vNa's.
Define the isometric embedding 
$$\iota_j : \mathcal{B}(\mathcal{H}_j) \to \mathcal{B}(\mathcal{H}_1 \oplus \dots \oplus \mathcal{H}_n),\quad x \mapsto ((\alpha_1, \dots, \alpha_n) \mapsto (0, \dots, 0, x\alpha_j, 0, \dots, 0)).$$
This map is the $n \times n$ bounded matrix where the $(j, j)$-th element is $x$ and the rest are zero.
Then 
$$M_1 \oplus \dots \oplus M_n := \linspan \{\iota_j (x)\ |\ j = 1, \dots, n,\ x \in M_j\}$$
is the direct sum of vNa's. If $n \geq 2$, then from 
$$Z(M_1 \oplus \dots \oplus M_n) = Z(M_1) \oplus \dots \oplus Z(M_n),$$
we deduce that $M_1 \oplus \dots \oplus M_n$ is not a factor.

\subsubsection{Tensor products}

The algebraic tensor product $\mathcal{B}(\mathcal{H}_1) \otimes \cdots \otimes \mathcal{B} (\mathcal{H}_n)$
acts on $\mathcal{H}_1 \overline{\otimes} \cdots \overline{\otimes} \mathcal{H}_n$ by
$$(x_1 \otimes \cdots \otimes x_n) (\alpha_1 \otimes \cdots \otimes \alpha_n) = (x_1 \alpha_1) \otimes \cdots \otimes (x_n \alpha_n)$$
for $x_j \in \mathcal{B} (\mathcal{H}_j)$ and $\alpha_j \in \mathcal{H}_j$, which implies 
$$\mathcal{B}(\mathcal{H}_1) {\otimes} \cdots {\otimes} \mathcal{B}(\mathcal{H}_n) \subseteq \mathcal{B}(\mathcal{H}_1 \overline{\otimes} \cdots \overline{\otimes} \mathcal{H}_n).$$
Finally, we define the tensor product of vNa's as 
$$M_1 \overline{\otimes} \cdots \overline{\otimes} M_n = (M_1 {\otimes} \cdots {\otimes} M_n)'' \cap \mathcal{B}(\mathcal{H}_1 \overline{\otimes} \cdots \overline{\otimes} \mathcal{H}_n).$$
 
\subsubsection{Compressions}

\begin{definition}
  Let $M \subseteq \mathcal{B}(\mathcal{H})$ be a vNa and $p \in \mathcal{B}(\mathcal{H})$ a projection.
  A compression of $M$ is $p M p = \{pxp\ |\ x \in M\}$.
  When $p \in M$, it is also called a corner.
\end{definition}

If $\mathcal{H} = \im p \oplus (\im p)^{\perp} = \im p \oplus \im (1 - p)$.
In this basis, elements of $p M p$ have the matrix form 
$$\begin{bmatrix}
  pxp & 0\\
  0 & 0
\end{bmatrix}.$$
If $M \ni p \neq 1$, then $pMp$ is a $*$-algebra and $pMp \subseteq M$
but it is not a subalgebra since $1_{M} = 1 _{\mathcal{B}(\mathcal{H})} \notin pMp$.
However, $pM p$ is a subalgebra of $\mathcal{B}(p\mathcal{H})$ with identity element $p$. 

\begin{definition}
  Let $\mathcal{K} \subseteq \mathcal{H}$ and $x \in \mathcal{B}(\mathcal{H})$.
  \begin{enumerate}
    \item $\mathcal{K}$ is invariant under $x$ if $x \mathcal{K} \subseteq \mathcal{K}$;
    \item $\mathcal{K}$ is reducing under $x$ if $\mathcal{K}$ is invariant under both $x$ and $x^*$.
  \end{enumerate}
  Now if $S \subseteq \mathcal{B}(\mathcal{H})$, then 
  \begin{enumerate}
    \item $\mathcal{K}$ is invariant under $S$ if $x \mathcal{K} \subseteq \mathcal{K}$ under all $x \in S$;
    \item $\mathcal{K}$ is reducing under $S$ if $\mathcal{K}$ is reducing under all $x \in S$.
  \end{enumerate}
\end{definition}

If $S \subseteq \bh$ is closed under $*$, then $\mathcal{K}$ is invariant under $S$ iff it is reducing under $S$. 
The following lemma was proved in the introductory course.

\begin{lemma}\label{lem:5.2}
  Let $\mathcal{K}^{\mathrm{closed}} \leq \mathcal{H}$ and $M \subseteq \bh$ an $*$-algebra.
  Let $p: \mathcal{H} \to \mathcal{K}$ be the orthogonal projection. Then $\mathcal{K}$ is reducing under $M$ iff $p \in M'$.
\end{lemma}

\begin{theorem}
  Let $M \subseteq \bh$ be a vNa and $p \in M$ a projection.
  Then $p M p$ and $M' p$ are vNa's in $\mathcal{B}(p \mathcal{H})$.
\end{theorem}

\begin{myproof}
  We will show that 
  $$(M' p)' \cap \mathcal{B} (p\mathcal{H}) = pM p,\quad (p M p)' \cap \mathcal{B} (p\mathcal{H}) = M' p.$$
  Then the bicommutant theorem will take care of the rest. 
  It is obvious that $(M' p)' \cap \mathcal{B} (p\mathcal{H}) \supseteq pM p$.
  For the converse, pick $x \in (M' p)' \cap \mathcal{B} (p\mathcal{H})$.
  Define $\widetilde{x} = xp = px  \in \mathcal{B}(\mathcal{H})$.
  For $y \in M'$, we have 
  $$y \widetilde{x} = ypx = xyp = xpy = \widetilde{x} y,$$
  which implies $\widetilde{x} \in M'' = M$.
  Then $x = pxp = p\widetilde{x}p \in pMp$.
  As before, $(p M p)' \cap \mathcal{B} (p\mathcal{H}) \supseteq M' p$
  is trivial and we just prove the converse. Take $y \in (pMp)' \cap \mathcal{B}(p\mathcal{H})$. Using continuous functional calculus, we can write $y$ as 
  a linear combinations of 4 unitaries. Since $pMp$ is closed under $*$,
  $(pM p)'$ is a vNa (and therefore a $C^*$-algebra). So we can assume w.l.o.g. that $y = u$ a unitary.
  Set $\mathcal{K} := \overline{M p \mathcal{H}}$ and let $q: \mathcal{H} \to \mathcal{K}$ be the orthogonal projection.
  Since $\mathcal{K}$ is reducing under $M$ and $M'$, which implies 
  $$q \in M' \cap M'' = M' \cap M = Z(M).$$
  Next, we extend $u$ to $\mathcal{K}$:
  $$\widetilde{u} (\sum_i \underbrace{x_i}_{\in M} p \underbrace{\alpha_i}_{\in \mathcal{H}}) = \sum_i x_i u p \alpha_i.$$
  We shall show that this is a well-defined isometry in $Mp\mathcal{H}$:
  \begin{align*}
    \| \widetilde{u} \sum_i x_i p \alpha_i \|^2 &= \sum_{i, j} \langle x_i u p \alpha_i, x_j u p \alpha_j \rangle\\
     &= \sum_{i, j} \langle(p x_j^* x_i p) u \alpha_i, u\alpha_j \rangle\\
     &= \sum_{i, j} \langle u p x_j^* x_i p \alpha_i, u\alpha_j \rangle\\ 
     &= \sum_{i, j} \langle p x_j^* x_i p \alpha_i, \alpha_j \rangle = \| \sum_i x_i p \alpha_i\|^2. 
  \end{align*}
  So $\widetilde{u}$ extends to an isometry on $\mathcal{K} = \overline{M p \mathcal{H}}$.
  By definition, $\widetilde{u}$ commutes with $M$ on $M p \mathcal{H}$, so also on $\mathcal{K}$.
  Thus for every $x \in M$ and $\alpha \in \mathcal{H}$, we have 
  $$x (\widetilde{u} q) \alpha = \widetilde{u} x q \alpha = (\widetilde{u} q)x \alpha,$$
  which implies $\widetilde{u} q \in M' \cap \bh$.
  Then $$\widetilde{u} qp \alpha = \widetilde{u} 1 p \alpha = 1 u p \alpha,$$
  which implies 
  $u = \widetilde{u} q p \in \bh$
  and $u \in M'p$. 
\end{myproof}

\begin{corollary}
  Suppose the vNa $M \subseteq \bh$ is a factor and let $p \in M$
  be a projection. Then $pMp$ and $M' p$ are factors (in $\mathcal{B}(p \mathcal{H})$).
\end{corollary}

\begin{myproof}
  Let $\mathcal{K} = \overline{Mp \mathcal{H}}$ and $q: \mathcal{H} \to \mathcal{K}$ the projection.
  From the previous proof, $q \in Z(M) = \C$. Then $q \in \{0, 1\}$. w.l.o.g. $p \neq 0$, so $q = 1$.
  Thus $\mathcal{K} = \mathcal{H}$, so $Mp\mathcal{H}$ is dense in $\mathcal{H}$.
  Consider $$\psi: M' \to M' p,\quad y \mapsto yp.$$
  We will prove that $\psi$ is an isomorphism of algebras. Obviously, it is additive.
  Since 
  $$\psi(xy) = xyp = xyp^2 = xpyp = \psi(x)\psi(y),$$
  it is also multiplicative. Same calculation shows $\psi(y^*) = \psi(y)^*$.
  Obviously, $\psi$ is surjective. Finally, we prove injectivity. Suppose $y \in M'$ satisfies $yp = 0$.
  Then for every $x \in M$ and $\alpha \in \mathcal{H}$, we get $yxp\alpha = x(yp)\alpha = 0.$
  Hence $y\big|_{Mp\mathcal{H}} = 0$, so by continuity, $y\big|_{\overline{Mp\mathcal{H}}} = y\big|_{\mathcal{K}} = 0$.
  But because $\mathcal{K} = \mathcal{H}$, this yields $y\big|_{\mathcal{H}} = 0$.
  As a result, we get 
  $$Z(M' p) = Z(M') p = \C \cdot p,$$ so $M' p$ is a factor.
  Similarly,
  $$Z(pMp) = (pMp) \cap (pMp)' = (M'p)' \cap M'p = Z(M'p) = \C p,$$
  so $pMp$ is a factor. 
\end{myproof}