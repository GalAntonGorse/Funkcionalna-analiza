\section{Bounded operators on Hilbert spaces}

\subsection{Polar decomposition}

Let $\mathcal{H}$ be a complex Hilbert space. Then $\bh$
is a $C^*$-algebra with the operator norm 
$$\|A\| = \sup_{\mu \in \mathcal{H}, \mu \neq 0} \frac{\| A\mu\|}{\| \mu\|} = \sup_{\mu \in \mathcal{H}, \|\mu\| = 1} \|A\mu\| = \sup_{\mu \in \mathcal{H}, \|\mu\| \leq 1} \|A\mu\|$$

\begin{remark}
  Recall that $A \in \bh$ is:
  \begin{enumerate}
    \item normal $\Leftrightarrow A^* A = A A^* \Leftrightarrow \|A\mu\| = \|A^* \mu\|,\ \forall \mu \in \mathcal{H}$;
    \item self-adjoint $\Leftrightarrow A^* = A \Leftrightarrow \langle A\mu, \mu\rangle \in \R,\ \forall \mu \in \mathcal{H}$;
    \item positive $\Leftrightarrow \textrm{$A = B^* B$ for some $B \in \mathcal{B}(\mathcal{H})$} \Leftrightarrow \langle A\mu, \mu\rangle \geq 0,\ \forall \mu \in \mathcal{H}$;
    \item isometry $\Leftrightarrow A^* A = I \Leftrightarrow \| A\mu \| = \| \mu\|,\ \forall \mu \in \mathcal{H}$;
    \item projection $\Leftrightarrow A^2 = A = A^* \Leftrightarrow \textrm{$A$ is an orthogonal projection onto some closed subspace of $\mathcal{H}$}$.
  \end{enumerate}
\end{remark}

\begin{lemma}
  An operator $A \in \bh$ is a partial isometry iff there exists a closed subspace $\mathcal{K} \leq \mathcal{H}$
  such that $A\big|_{\mathcal{K}}$ is an isometry and $A\big|_{\mathcal{K}^\perp} = 0$.
\end{lemma}

\begin{myproof}
  We first prove $(\Leftarrow)$. Obviously, $\mathcal{K}^\perp \subseteq \ker A$.
  From $Ax = 0$, where $x = y + z$ and $y \in \mathcal{K}, z \in \mathcal{K}^\perp$, we have 
  $$0 = Ax = A(y + z) = Ay + Az = Ay.$$
  But since $A\big|_{\mathcal{K}}$ is an isometry, $\| Ay\| = \|y\| = 0$, so $y = 0$ and $x \in \mathcal{K}^\perp$.
  Now we prove that $P = A^* A$ is the projection onto $\mathcal{K}$.
  For $x \in \mathcal{K}$, we have
  $$\langle Px, x \rangle = \langle A^* A x, x\rangle = \langle Ax, Ax\rangle = \|Ax\|^2 = \|x\|^2,$$
  so $$\|P\|  = \|A^* A\| \leq \|A\| \|A^*\| = \|A\|^2 = 1.$$
  From Cauchy-Schwartz:
  $$\langle Px, x\rangle \leq \|Px\| \|x\| \leq \|P\| \|x\|^2 \leq \|x\|^2.$$
  Since we have equality in Cauchy-Schwartz, there exists a $\lambda \in \C$ such that $Px = \lambda x$.
  But from $\langle Px, x\rangle = \|x\|^2$, it follows that $\lambda = 1$. So $P\big|_{\mathcal{K}} = \id$
  and for $x \in \mathcal{K}^\perp$, $Px = A^* Ax = 0$. Therefore, $P = A^* A$ is indeed a projection.
  Now onto the opposite direction $(\Rightarrow)$. Suppose $P = A^* A$ is a projection
  and denote $\mathcal{K} = \im P$. Since $\mathcal{K} = \ker (I - P)$, it is a closed subspace of $\mathcal{H}$.
  For $x \in \mathcal{K}$, we have 
  $$\| Ax\|^2 = \langle Ax , Ax\rangle = \langle Px, x\rangle = \langle x, x\rangle = \|x\|^2.$$
  But for $x \in \mathcal{K}^\perp$, we use the identity 
  $$(\im P)^\perp = \ker P^* = \ker P$$
  to get $Px = 0$, so $\| A x\|^2 = \langle Px, x\rangle = 0$ and $\| Ax\| = 0$.
\end{myproof}

\begin{theorem}[Polar decomposition]
  Let $\mathcal{H}$ be a Hilbert space and $x \in \bh$.
  Then there exists a partial isometry $v$ such that $x = v \cdot |x|$ and 
  $\ker v = \ker |x| = \ker x$. This decomposition is unique: if $x = wy$
  for $y \geq 0$ and partial isometry $w$ such that $\ker y = \ker w$, then $w = v$ and $y = |x|$. 
\end{theorem}

\begin{myproof}
  First we prove the existence. Define $$v_0 : \im |x| \to \im x,\quad |x|y \mapsto xy.$$
  Since \begin{align*}
    \| |x| y\|^2 &= \langle |x|y, |x|y \rangle \\
    &= \langle |x|^2 y, y\rangle\\
    &= \langle x^* x y, y\rangle\\
    &= \langle xy, xy\rangle\\
    &= \| xy\|^2.
  \end{align*}
  The above $v_0$ is well defined. It is also linear and isometric.
  By continuity, extend $v_0$ to a map $\overline{\im |x|} \to \overline{\im x}$.
  Now $v_0$ can be extended to $v: \mathcal{H} \to \mathcal{H}$ by setting 
  $v\big|_{(\im |x|)^\perp} = 0.$ By previous lemma, $v$ is a partial isometry.
  By definition, $x = v \cdot |x|$ and $\ker v = (\im |x|)^\perp = \ker |x| = \ker x$.
  Next, we prove uniqueness. If $x = wy$ as in the statement, then $\ker w = \ker y = (\im y)^\perp$,
  so $w$ is a partial isometry on $\overline{\im y}$. From there, we get 
  $$|x|^2 = (wy)^* (wy) = y^* w^* w y = y^* y = y^2,$$
  which implies 
  $$|x| = (|x|^2)^{\frac{1}{2}} = (y^2)^{\frac{1}{2}} = y.$$
  Now $$w|x| \mu = wy \mu = x\mu$$
  together with 
  $$\ker w = (\im y)^\perp = (\im |x|)^\perp$$ implies $w = v$.
\end{myproof}

Now we can also prove the statement in the example \ref{ex:1.4}.

\begin{proposition}
  The extreme points of the unit ball of $\bh$ are exactly the elements $V \in \bh$ such that 
  $$(1 - V V^*) \bh (1 - V^* V) = 0.$$ In particular, $V^* V$ and $V V^*$ are projections.
\end{proposition}

\begin{myproof}
  Let $V \in A$ be an extreme point of the unit ball of $\bh$, so $\sigma (V) \subseteq [-1, 1]$. Write 
  $$V = \frac{1}{2} V (2 - |V|) + \frac{1}{2} V |V|.$$
  Since the functions $z \mapsto z (2 - |z|)$ and $z \mapsto |z| (2 - |z|)$ coincide and are both bounded above by $1$ on $\sigma(x)$,
  we have $\| V (2 - |V|)\| = \| |V| (2 - |V|)\| \leq 1$ by continuous functional calculus.
  This implies that $V (2 - |V|)$ is in the unit ball of $\bh$. The same can be said about $V |V|$ by the same argument.
  Now since $V$ is an extreme point, we must have $V = V |V|$. Multiplying on the left with $V^*$, 
  we get $|V|^2 = |V|^3$. This means that the functions $z \mapsto z^2$ and $z \mapsto z^3$ coincide on $\sigma (|V|)$,
  which implies that $\sigma (|V|) \subseteq \{0, 1\}$. As a result, $|V|$ is a projection, so $P := |V| = |V|^2 = V^* V$.
  The same can be said about $Q := V V^*$, since we know that $\sigma (V^* V) \setminus \{0\} = \sigma (V V^*) \setminus \{0\}$. 
  This means that $V$ is a partial isometry. By the previous lemma, $P$ is a projection onto the initial space of 
  $V$, so $QV = VP = V$. Now suppose $W := (1 - Q) Z (1 - P) \neq 0$ for some $Z$ in the unit ball.
  Then 
  \begin{align*}
     \| V + W\|^2 &= \| QVP + (1 - Q)Z(1 - P) \|^2\\
     &= \|( QVP + (1 - Q)Z(1 - P))^* ( QVP + (1 - Q)Z(1 - P))\|\\
     &= \| (P V^* Q + (1 - P)Z^*(1 - Q)) (QVP + (1 - Q)Z(1 - P))\|\\
     &= \| P V^* Q V P + (1 - P) Z^* (1 - Q) Z (1 - P)\|\\
     &= \| P V^* V P + (1 - P) W^* W (1 - P)\|\\
     &= \max \{\|V^* V\|, \|W^* W\|\}\\
     &= \max \{\|V\|^2, \| W\|^2\} = 1
  \end{align*}
  and similarly $\| V - W\|^2 = 1$. Therefore we have a decomposition 
  $$V = \frac{1}{2} (V + W) + \frac{1}{2} (V - W)$$
  and $V$ is not an extreme point, leading to a contradiction. Conversely, suppose that 
  $(1 - V V^*) \bh (1 - V^* V) = 0.$ Then we have 
  $$0 = V^* (1 - V V^*) V (1 - V^* V) = V^* V (1 - V^* V)^2.$$
  This implies that the function $z \mapsto z( 1 - z)^2$ must be zero on $\sigma (V^* V)$, so 
  $\sigma (V^* V) \subseteq \{0, 1\}$ and $P := V^* V$ is a projection. By the same argument,
  $$0 = (1 - V V^*) V (1 - V^* V)V^* =(1 - V V^*)^2  V V^*$$ and $Q := V V^*$ is a projection as well.
  Assume that $V = \frac{1}{2} U + \frac{1}{2} W$ for $U, W$ in the unit ball. 
  Again we have $V = VP = QV$, so $$V = \frac{1}{2} UP + \frac{1}{2} WP$$ and 
  \begin{align*}
    4 P &= 4 V^* V = P U^* U P + PW^* WP + P U^* W P + P W^* U P\\
    &= 2 (P U^* U P + PW^* WP) - P (U - W)^* (U - W) P\\
    &\leq 4P - P (U - W)^* (U - W) P.
  \end{align*} 
  This immediately implies that $(U - W)P = 0$. Similarly, we have $ Q(U - W) = 0.$ Now 
  \begin{align*}
    U - W &= Q (U - W) P + (1 - Q) (U - W) P \\
    &+ Q (U - W) (1 - P) + (1 - Q) (U - W) (1 - P) = 0. \qedhere
  \end{align*}
\end{myproof}

\begin{remark}
  The above theorem holds for any $C^*$-algebra, not just $\bh$.
  We can identify any general $C^*$-algebra with an algebra of operators on some Hilbert space
  and then the above proof carries over verbatim. 
\end{remark}

\begin{corollary}\label{cor:4.1}
  $$\ext (\bh)_1 = \{V \in \bh\ |\ \textrm{$V$ or $V^*$ is an isometry}\}.$$
\end{corollary}

\begin{myproof}
  We need to prove the inclusion $(\subseteq)$. If $V$ is an extreme point, then $V^* V$ and $V V^*$ are projections
  Therefore, $V$ is a partial isometry with the initial space $(\ker V)^\perp$ and 
  $V^*$ is a partial isometry with the initial space $(\ker V^*)^\perp$.
  Assume neither $V$ nor $V^*$ are full isometries, so their initial spaces are proper subspaces of $\mathcal{H}$.
  This means that there exist vectors $0 \neq x \in \ker V$ and $0 \neq y \in \ker V^*$.
  Define $P$ as a rank-one projection from $x$ to $y$. Then
  $$(1 - V V^*) P (1 - V^* V) x = y \neq 0,$$
  so $(1 - V V^*) P (1 - V^* V) \neq 0$, contradiction.
\end{myproof}

\subsection{Trace class operators}

\begin{definition}
  Let $X, Y$ be Banach spaces.
  An operator $A \in \mathcal{B} (X, Y)$ has \emph{finite rank} if $\rank A := \dim \overline{\im A} < \infty$.
  The set of finite rank operators is denoted by $\mathcal{F} (X, Y)$.
  We also denote $\mathcal{F} (X) := \mathcal{F} (X, X)$.
\end{definition}

\begin{remark}
  Let $A \in \bh$, where $\mathcal{H}$ is a Hilbert space.
  We know that 
$$\im A^* = \im (A^* |_{(\ker A^*)^\perp}) = \im (A^* \big|_{\overline{\im A}}).$$
From there, we can conclude that $\rank A < \infty$ iff $\rank A^* < \infty$.
\end{remark}

If $\alpha, \beta \in \mathcal{H}$, then we can define the operator
$$\alpha \otimes \overline{\beta} : \mathcal{H} \to \mathcal{H},\quad y \mapsto \langle y, \beta \rangle \cdot \alpha.$$
It is trivial to see that $\rank (\alpha \otimes \overline{\beta}) \leq 1$ and $(\alpha \otimes \overline{\beta})^* = \beta \otimes \overline{\alpha}$.
By Riesz's representation theorem, we also know that every rank-one operator on $\mathcal{H}$
is of this form.
If $\| \alpha\| = \| \beta \| = 1$, then $\alpha \otimes \overline{\beta}$
is a partial isometry with initial space $\C \beta$ and image $\C \alpha$.
Then $$\mathcal{F} (\mathcal{H}) = \mathrm{span}\ \{\alpha \otimes \overline{\beta}\ |\ \alpha, \beta \in \mathcal{H}\}.$$
For $x, y \in \bh$ we have $$x (\alpha \otimes \overline{\beta}) y = (x \alpha) \otimes \overline{(y^* \beta)}.$$  
\begin{lemma}\label{lem:4.1}
  Let $x \in \bh$ have the polar decomposition $x = v \cdot |x|$.
  Then for all $y \in \mathcal{H}$, we have 
  $$2 \left| \langle xy , y \rangle \right| \leq \langle |x| y, y\rangle + \langle |x|v^* y, v^* y \rangle.$$
\end{lemma}

\begin{myproof}
  Let $\lambda \in \mathbb{T}$. Then 
  \begin{align*}
    0 &\leq \|(|x|^{\frac{1}{2}} - \lambda |x|^{\frac{1}{2}} v^*) y\|^2\\
    &= \| |x|^{\frac{1}{2}} y\|^2 - 2 \real \overline{\lambda} \langle |x|^{\frac{1}{2}} y, |x|^{\frac{1}{2}} v^* y\rangle + \||x|^{\frac{1}{2}} v^* y\|^2.
  \end{align*}
  Now pick $\lambda$ such that $\overline{\lambda} \langle |x|^{\frac{1}{2}} y, |x|^{\frac{1}{2}} v^* y \rangle \geq 0$ and we are done.
\end{myproof}

\begin{definition}
  Let $(e_i)_{i \in I}$ be an orthonormal basis for $\mathcal{H}$. For $x \in \mathcal{B}(\mathcal{H})_+$,
  define the \emph{trace} 
  $$\trace (x) = \sum_{i \in I} \langle x e_i, e_i \rangle \in [0, \infty].$$
  We call $x \in \mathcal{B}(\mathcal{H})$ \emph{trace class} if 
  $$\| x\|_1 := \trace (|x|) < \infty.$$
  The set of trace class operators on $\mathcal{H}$ will be denoted by $L^1 (\bh, \trace)$.
\end{definition}

\begin{remark}
  Let $\{h_i\ |\ i \in I\} \subseteq \mathcal{H}$ be a set of vectors in a Hilbert space. 
  We already know that the collection of finite sets $F \subseteq I$ forms a directed set.
  Then vectors $h_F := \sum_{i \in F} h_i$ form a net in $\mathcal{H}$.
  We define $\sum_{i \in I} h_i$ as the limit of the net $(h_F)$, if it exists.
  Note that if $I$ is countable, this definition of a convergent sum does not necessarily
  coincide with the usual one. In other words, for a set $\{h_n\ |\ n \in \N\}$ in a Hilbert space $\mathcal{H}$, the convergence of a sum
  $\sum_{n \in \N} h_n$ is not equivalent to the convergence of a sum $\sum_{n = 1} ^\infty h_n$ -- in fact, the convergence of a former sum implies the convergence of the latter one (with the sums being equal).
  The converse holds if $\sum_{n = 1} ^\infty \| h_n\| < \infty$.
\end{remark}

\begin{remark}
  If $x \in \bh_+$ and $\trace(x) = \sum_{i \in I} \langle xe_i, e_i\rangle < \infty$, then $\langle xe_i, e_i\rangle > 0$ holds for at most 
  countably many $e_i$. Let $(e_n)_{n \in \N}$ be a set of such basis vectors.
  Then $\sum_{i \in I} \langle x e_i, e_i\rangle = \sum_{n = 1} ^\infty \langle xe_n, e_n\rangle$.
\end{remark}

\begin{lemma}\label{lem:4.2}
  For all $x \in \mathcal{B}(\mathcal{H})$ we have $\trace (x^* x) = \trace(x x^*)$.
\end{lemma}

\begin{myproof}
  \begin{align*}
    \trace (x^* x) &= \sum_i \langle x^* x e_i , e_i\rangle = \sum_i \langle x e_i, x e_i \rangle\\
    &= \sum_i \| x e_i\|^2 = \sum_i \sum_j \langle x e_i, e_j \rangle \overline{\langle x e_i, e_j \rangle}\\
    &= \sum_j \sum_i \langle e_i, x^* e_j \rangle \overline{\langle e_i, x^* e_j \rangle} = \sum_j \sum_i \langle x^* e_j, e_i\rangle \overline{\langle x^* e_j , e_i\rangle}\\
    &= \sum_j \| x^* e_j \|^2 = \sum_j \langle x^* e_j, x^* e_j\rangle\\
    &= \sum_j \langle x x^* e_j, e_j \rangle = \trace(x x^*) \qedhere
  \end{align*}
\end{myproof}

\begin{corollary}
  If $x \in \bh_+$ and $u \in \mathcal{U} (\mathcal{H})$, then 
  $$\trace (u^* x u) = \trace (x).$$
  In particular, the trace of a positive operator is independent of the choice of the 
  orthonormal basis for $\mathcal{H}$.
\end{corollary}

\begin{myproof}
  Since $x \in \bh_+$, there exists a $y \in \bh$
  such that $x = y^* y$. By lemma \ref{lem:4.2}, we have 
  \begin{align*}
    \trace (x) &= \trace (y^* y) = \trace (y y^*)\\
    &= \trace (u^* y^* y u) = \trace (u^* x u).
  \end{align*}
  If $(f_i)$ is another ONB for $\mathcal{H}$, then there exists $u \in \mathcal{U}(\mathcal{H})$ such that $u e_i = f_i$ for all indices $i$:
  \begin{align*}
    \sum_i \langle x f_i, f_i \rangle &= \sum_i \langle x u e_i, u e_i\rangle\\
    &= \sum_i \langle u^* x u e_i, e_i \rangle\\
    &= \trace (u^* x u) = \trace (x). \qedhere
  \end{align*}
\end{myproof}

\begin{definition}
  If $(e_i)$ is ONB for $\mathcal{H}$ and $x \in L^1 (\bh)$, then its trace is 
  $$\trace(x) := \sum_{i \in I} \langle x e_i, e_i \rangle.$$
\end{definition}

By lemma \ref{lem:4.1} and the proof below, we get
\begin{align*}
  2 | \trace(x) | &\leq \sum_{i \in I} 2|\langle xe_i, e_i\rangle|\\
  &\leq \sum_{i \in I} \langle |x|e_i, e_i\rangle + \langle |x| v^* e_i, v^* e_i\rangle\\
  &= \trace (|x|) + \trace (v |x| v^*) \\
  &\leq \|x\|_1 + \|x\|_1\\
  &= 2\| x\|_1.
\end{align*}

\begin{theorem}
  \begin{enumerate}
    \item $L^1 (\bh)$ is a two-sided ideal in $\mathcal{B}(\mathcal{H})$
    that is closed under involution. 
    \item $L^1 (\bh)$ is a linear span of all positive operators of finite trace.
    \item Trace is independent of the ONB and $\| \cdot \|_1$ is a norm on $L^1 (\bh)$.
  \end{enumerate}
\end{theorem}

\begin{myproof}
  Let $A, B \in L^1 (\bh)$ and satisfy the polar decompositions:
  $$A + B = U |A + B|,\quad A = V |A|,\quad B = W |B|.$$
  Let $(e_i)$ be an ONB. Then 
  \begin{align*}
    \sum_{i = 1} ^N \langle |A + B| e_i, e_i \rangle &= \sum_{n = 1} ^N |\langle U^* (A + B) e_n, e_n\rangle|\\
    &\leq \sum_{n = 1} ^N |\langle U^* A e_n, e_n \rangle| + \sum_{n = 1} ^N |\langle U^* B e_n, e_n \rangle| \\
    &= \sum_{n = 1} ^N |\langle U^* V |A| e_n, e_n \rangle| + \sum_{n = 1} ^N |\langle U^*W |B| e_n, e_n \rangle|.
  \end{align*}
  We can bound the first term:
  \begin{align*}
    \sum_{n = 1} ^N |\langle U^* V |A| e_n, e_n \rangle| &= \sum_{n = 1} ^N |\langle |A|^{\frac{1}{2}} e_n, |A|^{\frac{1}{2}} V^* U e_n \rangle|\\
    &\leq \sum_{n = 1} ^N \||A|^{\frac{1}{2}} e_n\| \||A|^{\frac{1}{2}} V^* U e_n\|\\
    &\leq \left(\sum_{n = 1} ^N \||A|^{\frac{1}{2}} e_n\|^2 \right)^{\frac{1}{2}} \left(\sum_{n = 1} ^N \||A|^{\frac{1}{2}} V^* U e_n\|^2\right)^{\frac{1}{2}}.
  \end{align*}
  Since $\| |A|^{\frac{1}{2}} e_n \|^2 = \langle |A|^{\frac{1}{2}} e_n, |A|^{\frac{1}{2}} e_n \rangle = \langle |A| e_n, e_n\rangle$, the expression in the first bracket goes to $\trace |A|$.
  Next, we prove that the expression in the second bracket is less or equal to $\trace |A|$:
  $$\sum_{n = 1} ^N \langle |A|^{\frac{1}{2}} V^* U e_n, |A|^{\frac{1}{2}} V^* U e_n \rangle = \sum_{n = 1} ^N \langle U^* V |A| V^* U e_n, e_n \rangle \xrightarrow{N \to \infty} \trace |A|.$$
  Pick an ONB for $\mathcal{H}$ as follows: each $f_j$ should be in $\ker U$ or $(\ker U)^\perp$.
  Then $$\trace (U^* V |A| V^* U) \leq \trace (V |A| V^*).$$
  By similar argument, 
  $$\trace (V |A| V^*) \leq \trace(|A|)$$
  and we are done:
  $$\sum_{n = 1} ^N |\langle U^* V |A| e_n, e_n \rangle| \leq \trace |A|.$$
  Similarly, 
  $$\sum_{n = 1} ^N |\langle U^* W |B| e_n, e_n \rangle| \leq \trace |B|,$$
  which implies 
  $\trace |A + B| \leq \trace |A| + \trace |B|$.
  We have proved that $L^1 (\bh)$ is a vector space and $\| \cdot \|_1$ is a norm.
  Clearly, $L^1 (\bh)$ contains all positive operators with finite trace,
  so also their linear span. Next we prove that it is a two-sided ideal of $\bh$.
  Let $A \in L^1 (\bh)$ and $B \in \bh$.
  Since every operator is a linear combination of four unitaries, we can assume w.l.o.g.~that $B = U$ is a unitary.
  Then 
  $$|UA| = (A^* U^* U A)^{\frac{1}{2}} = (A^* A)^{\frac{1}{2}} = |A|,$$
  so $BA = UA \in L^1 (\bh)$. Furthermore,
  $$|AU| = (U^* A^* A U)^{\frac{1}{2}} = U^* |A| U,$$
  which implies 
  $$\trace |AU| = \trace (U^* |A| U) = \trace |A|$$
  and $AB = AU \in L^1 (\bh)$.
  Now we prove that $L^1 (\bh)$ is closed under involution.
  Let $A = U |A|$ and $A^* = V |A^*|$ be polar decompositions. Then 
  $$|A^*| = V^* A^* = V^* (U |A|)^* = V^* |A| U^*.$$
  If $A \in L^1 (\bh)$, then $|A| \in L^1 (\bh)$, so 
  $$|A^*| = V^* |A| U^* \in L^1 (\bh).$$ This gives us $A^* \in L^1 (\bh)$.
  Finally, we prove that $L^1 (\bh)$ is the linear span of all positive operators of finite trace.
  Let $x \in L^1 (\bh)$ and $a \in \bh$. The following polarization identity holds:
  $$4 a |x| = \sum_{k = 0} ^3 i^k \underbrace{(a + i^k) |x| (a + i^k)^*}_{\textrm{positive and finite trace}}.$$
  If $a = v$ partial isometry from the polar decomposition theorem, then 
  $$x = v |x| = \sum_{k = 0} ^3 \frac{i^k}{4} {(v + i^k) |x| (v + i^k)^*}.$$
  is a linear combination of four positive operators with finite trace.
\end{myproof}

\begin{proposition}
  Let $x \in L^1 (\bh)$ and $a, b \in \bh$. Then 
  \begin{itemize}
    \item $\| x\| \leq \|x\|_1$;
    \item $\| axb\|_1 \leq \| a\| \|b\| \|x\|_1$;
    \item $\trace (ax) = \trace (xa)$.
  \end{itemize}
\end{proposition}

\begin{myproof}
  \begin{enumerate}
    \item
    \begin{align*}
      \| x\| &= \| |x|\| = \| |x|^{\frac{1}{2}} \|^2\\
      &= \sup _{\| \alpha\| = 1} \langle |x|^{\frac{1}{2}} \alpha, |x|^{\frac{1}{2}} \alpha \rangle = \sup_{\| \alpha \| = 1} \langle |x| \alpha, \alpha \rangle \\
      &\leq \trace |x| = \| x\|_1.
    \end{align*}
    \item We begin with 
    \begin{equation*}
      | a x |^2 = x^* a ^* a  x \leq \| a ^* a \| x^* x = \| a^* a\|\cdot |x|^2 = \|a\|^2\cdot |x|^2
    \end{equation*}
    and since $|ax| \leq \| a\| \cdot |x|$ we get $\| ax\|_1 \leq \|a\| \cdot \|x\|_1$.
    But $\|x\|_1 = \|x^*\|_1$, so we also get $\| xb\|_1 \leq \|b\| \cdot \|x\|_1$.
    \item Since every element of $\bh$ is a linear combination of 4 unitaries, we can w.l.o.g.~assume $a = u \in \mathcal{U} (\mathcal{H})$. Then 
    \begin{align*}
      \trace (xu) &= \sum_i \langle x u e_i, e_i \rangle = \sum_i \langle x u e_i, u^* u e_i\rangle\\
      &= \sum_i \langle ux ue_i, u e_i \rangle = \trace (ux). \qedhere
    \end{align*}
  \end{enumerate}
\end{myproof}

\begin{remark}
  We have the following identities:
  \begin{enumerate}
    \item $\trace (\alpha \otimes \overline{\beta}) = \langle \alpha, \beta \rangle$;
    \item $\mathcal{F} (\mathcal{H})$ is dense in $(L^1 (\bh), \| \cdot \|_1)$.
  \end{enumerate}
\end{remark}

\begin{theorem}
  $(L^1 (\bh), \| \cdot \|_1)$ is a Banach space.
\end{theorem}

\begin{myproof}
  We only have to prove completeness. Let $(x_n)_n$ be a Cauchy sequence in $(L^1 (\bh), \| \|_1)$.
  Since $\| \cdot\| \leq \| \cdot \|_1$, $(x_n)$ is a Cauchy sequence in $(\mathcal{B}(\mathcal{H}), \| \|)$.
  But $(\mathcal{B}(\mathcal{H}), \| \|)$ is a Banach space, so there exists $x \in \bh$ such that $x_n \to x$ in norm-topology.
  Notice that 
  $$x^* x - x_n^* x_n = x^* (x - x_n) + (x - x_n)^* x_n.$$
  By continuity of the continuous functional calculus, this implies $|x_n| \to |x|$, meaning that $\| |x_n| - |x| \| \to 0$.
  Next we prove that $x \in L^1 (\bh)$. For any ONB $(e_i)_i$, we have 
  $$\sum_{i = 1} ^k \langle |x| e_i, e_i \rangle = \lim_{n \to \infty} \sum_{i = 1} ^k \langle |x_n| e_i, e_i \rangle \leq \lim_{n \to \infty} \trace |x_n| = \lim_{n \to \infty} \|x_n\|_1 < \infty.$$
  Here, we used the fact that $\| x_n - x_k \|_1 \geq \| x_n \|_1 - \|x_k\|_1$, so the sequence $(\| x_n\|_1)_n$ is Cauchy and therefore has a limit.
  This proves that $x \in L^1 (\bh)$ and $\| x\|_1 \leq \lim_{n \to \infty} \| x_n\|_1$.
  Finally, we have to show that $\| x_n - x\|_1 \to 0$. Let $\varepsilon > 0$. Pick $N \in \N$ such that for every $n > N$, we get $\| x_n - x_N \|_1 < \frac{\varepsilon}{3}$.
  Let $\mathcal{H}_0 \subseteq \mathcal{H}$ be a finite dimensional subspace such that 
  $$\| x_N P_{\mathcal{H}_0 ^\perp} \|_1,\ \| x P_{\mathcal{H}_0 ^\perp} \|_1 < \frac{\varepsilon}{3}.$$
  Then for every $n > N$, we get that 
  \begin{align*}
    \| x - x_n \|_1 &\leq \| (x - x_n) P_{\mathcal{H}_0} \|_1 + \| (x - x_n) P_{\mathcal{H}_0 ^\perp} \|_1\\
    &\leq \| (x - x_n) P_{\mathcal{H}_0} \|_1 + \| x P_{\mathcal{H}_0 ^\perp} - x_N P_{\mathcal{H}_0 ^\perp}\|_1 + \| x_N P_{\mathcal{H}_0 ^\perp} - x_n P_{\mathcal{H}_0 ^\perp}\|_1\\
    &\leq \| (x - x_n) P_{\mathcal{H}_0 }\|_1 + \frac{\varepsilon}{3} + \frac{\varepsilon}{3} + \|x_N - x_n\|_1 \| P_{\mathcal{H}_0 ^\perp}\|\\
    &< \| (x - x_n) P_{\mathcal{H}_0 }\|_1 + \varepsilon\\
    &\leq \| (x - x_n) \|  \| P_{\mathcal{H}_0 }\|_1 + \varepsilon \xrightarrow[n \to \infty]{} \varepsilon.
  \end{align*}
  Since $\varepsilon > 0$ was arbitrary, this shows $x_n \xrightarrow[\| \cdot \|_1]{} x$.
\end{myproof}

\begin{theorem}\label{thm:4.1}
  The map 
  $$\Psi: \bh  \to L^1 (\bh)^*,\quad a \mapsto (\psi_a: x \mapsto \trace (ax))$$
  is an isometric isomorphism of Banach spaces.
\end{theorem}

\begin{myproof}
  We notice that $\Psi$ is linear and a contraction because the norms $\|\cdot \|$
  and $\| \cdot \|_1$ are comparable. We will first show that $\Psi$ is surjective.
  Let $\varphi \in L^1 (\bh)^*$. Notice that 
  $$(\alpha, \beta) \mapsto \varphi(\alpha \otimes \overline{\beta})$$
  is a bounded sesquilinear form in $\mathcal{H}$. By the introductory course, there exists 
  an $a \in \bh$ such that 
  $$\varphi(\alpha \otimes \overline{\beta}) = \langle a \alpha, \beta \rangle = \trace (a\alpha \otimes \overline{\beta}) = \trace (a(\alpha \otimes \overline{\beta})) = \psi_a (\alpha \otimes \overline{\beta}).$$
  So $\varphi$ and $\psi_a$ agree on $\mathcal{F} (\mathcal{H})$, so by bounded density $\varphi = \psi_a$.
  Finally, 
  $$\| a\| = \sup_{\alpha, \beta \in (\mathcal{H})_1} |\langle a \alpha, \beta \rangle| = \sup_{\alpha, \beta \in (\mathcal{H})_1} |\trace (a (\alpha \otimes \overline{\beta})) | \leq \| \psi_a \|_1.$$
  But since 
  $$\| \psi_a \|_1 = \sup_{x \in (L^1(\bh))_1} |\trace (ax)| = \sup_{x \in (L^1(\bh))_1} \| ax \|_1 \leq \sup_{x \in (L^1(\bh))_1} \| a \| \|x \|_1 = \| a\|,$$
  we have $\| a\| = \| \psi_a\|_1$ and $\psi$ is isometric.
\end{myproof}

\begin{corollary}
  The map $$\Phi: L^1 (\bh) \to \mathcal{K} (\mathcal{H})^*, \quad x \mapsto (\varphi_x: a \mapsto \trace (ax))$$
  is an isometric isomorphism of Banach spaces.
\end{corollary}

\begin{proof}
  Same as that of theorem \ref{thm:4.1}.
\end{proof}

\begin{definition}
  Let $X, Y$ be Banach spaces. An operator $T\in \mathcal{B}(X, Y)$ is said to be \emph{compact}
  if $\overline{T((X)_1)}$ is compact. The space of compact operators is $\mathcal{K} (X, Y)$.
  We also write $\mathcal{K} (X) := \mathcal{K} (X, X)$.
\end{definition}

From the introductory course, we know the following statements about compact operators.

\begin{proposition}
  Let $T \in \mathcal{B}(X, Y)$. The following are equivalent.
  \begin{enumerate}
    \item $T$ is compact;
    \item $T$ maps bounded maps in $X$ into relatively compact maps in $Y$;
    \item $T$ maps bounded sequences in $X$ into sequences in $Y$ that have an accumulation point.
  \end{enumerate}
  If $X, Y$ are Hilbert spaces, then this is also equivalent to the following.
  \begin{enumerate}\setcounter{enumi}{3}
    \item $T \in \overline{\mathcal{F}(X, Y)}$.
  \end{enumerate}
\end{proposition}

\begin{remark}
  $\mathcal{K}(\mathcal{H})$ is a closed ideal in $\bh$.
\end{remark}

\begin{theorem}[Singular value decomposition]
  For $K \in \mathcal{K} (\mathcal{H})$, there exists orthonormal bases $(e_i)_i$ and $(f_j)_j$ for $\mathcal{H}$
  and $\sigma_1 \geq \sigma_2 \geq \dots \geq 0$ such that 
  \begin{equation}\label{eq:4.1}
    K x = \sum_{n = 1} ^\infty \sigma_n \langle x, e_n \rangle f_n.
  \end{equation}
  As a result, $$|K| x = \sum \sigma_n \langle x, e_n \rangle e_n.$$  
\end{theorem}

\begin{proposition}
  Let $X$ be a Banach space. Then the following statements are equivalent:
  \begin{enumerate}
    \item $\id: X \to X$ is compact;
    \item $(X)_1$ is compact;
    \item $\dim X < \infty$.
  \end{enumerate}
  The equivalence of the last two items is also known as the \emph{Riesz lemma}.
\end{proposition}

\begin{theorem}
  \begin{enumerate}
    \item $L^1 (\bh) \subseteq K(\mathcal{H})$.
    \item $K \in \mathcal{H}$ is a $L^1 (\bh)$ iff $\sum_{k = 1} ^\infty \sigma_n < \infty$.
  \end{enumerate}
\end{theorem}

\begin{myproof}
  \begin{enumerate}
    \item If $x \in L^1 (\bh)$ then there exists $(x_n)_n$ in $\mathcal{F} (\mathcal{H})$ such that $\|x_n - x\|_1 \to 0$.
    Since $\| \cdot \| \leq \| \cdot \|_1$, we get $\|x_n - x \| \to 0$ and $x \in \overline{(\mathcal{F}, \| \cdot\|)} = \mathcal{K} (\mathcal{H})$.
    \item This follows from $\trace |K| = \sum \sigma_n$ for $K$ as in \eqref{eq:4.1}. \qedhere
  \end{enumerate}
\end{myproof}

\subsection{Hilbert--Schmidt operators}

\begin{definition}
  An element $x \in \bh$ is a \emph{Hilbert--Schmidt operator} 
  if $$|x|^2 = x^* x \in L^1 (\bh).$$
  The set of all such elements is denoted by $L^2 (\bh, \trace)$. 
\end{definition}

\begin{proposition}
  \begin{enumerate}
    \item $L^2 (\bh) \lhd \bh$ and is closed under $*$.
    \item If $x, y \in L^2 (\bh)$, then $xy, yx \in L^1 (\bh)$ and $\trace(xy) = \trace (yx)$.
  \end{enumerate}
\end{proposition}

\begin{remark}
  Beware: there exist $a, b \in \bh$ such that $ab \in L^1 (\bh)$
  and $ba \notin L^1 (\bh)$. However, if $ab, ba \in L^1 (\bh)$,
  then $\trace (ab) = \trace (ba)$.
\end{remark}

\begin{myproof}
  For $\alpha \in \C$ and $x, y \in \mathcal{B}(\mathcal{H})$, we have $|\alpha x|^2 = |\alpha|^2 |x|^2$.
  Similarly, $|x + y|^2 \leq |x + y|^2 + |x - y|^2 = 2 (|x|^2 + |y|^2)$, so $L^2 (\bh)$ is a complex vector space.
  Since $|ax|^2 \leq \| a\|^2 \cdot |x|^2$, we have $L^2 (\bh)$ is a left ideal of $\bh$.
  From 
  $$\trace |x|^2 = \trace (x^* x) = \trace (x x^*) = \trace |x^* |^2,$$
  we deduce that $L^2 (\bh)$ is closed under involution.
  If $x \in L^2 (\bh)$ and $b \in \bh$,
  then $x^* \in L^2 (\bh)$, which implies $b^* x^* \in L^2 (\bh)$ and finally $xb = (b^* x^*)^* \in L^2 (\bh)$,
  so $L^2 (\bh) \lhd \bh$.
  Next, we use the polarization identity 
  $$4 y^* x = \sum_{k = 0} ^3 i^k |x + i^k y|^2.$$
  If $x, y \in L^2 (\bh)$, then this shows $y^* x \in L^1 (\bh)$
  and 
  \begin{align*}
    4 \trace (y^* x) &= \sum_{k = 0} ^3 i^k \trace ((x + i^k y)^*(x + i^k y))\\
    &= \sum_{k = 0} ^3 i^k \trace ((x + i^k y)(x + i^k y)^*)\\
    &= 4 \trace (x y^*). \qedhere
  \end{align*}
\end{myproof}

  On $L^2 (\bh)$ we have the sesquilinear form $\langle x, y\rangle_2 := \trace (y^* x)$.
  It is well-defined and positive definite, so it is a scalar product. The induced norm is denoted by $\| \cdot \|_2$.
  For every $y \in L^2 (\bh)$, we have 
  $$\| y\| = \| y^* y\|^{\frac{1}{2}} \leq \| y^* y\|_1 ^{\frac{1}{2}} = \|y\|_2.$$
  Similarly, we have $$\| axb\|_2 = \| a\| \cdot \|x\|_2 \cdot \| b\|$$ for all $x \in L^2 (\bh)$
  and $a, b \in \bh$. As before, $\mathcal{F} (\mathcal{H})$ are dense in $L^2 (\bh)$ with respect to $\| \cdot \|_2$ and $L^2 (\mathcal{B}(\mathcal{H})) \subseteq \mathcal{K} (\mathcal{H})$.
  Using singular values $(\sigma_n)_n$ of a compact $K \in \mathcal{K} (\mathcal{H})$, we have $K \in L^2 (\bh)$
  iff $\sum_{k = 0} ^\infty \sigma_j ^2 < \infty$.
  For every $x \in L^1 (\bh)$, we have 
  $$\| x\|_2 = \sup_{y \in L^2 (\mathcal{B}(\mathcal{H})),\ \| y\|_2 = 1} | \trace (y^* x) | \leq \sup_{y \in L^2 (\bh),\ \| y\|_2 = 1} \|y\| \cdot \| x\|_1 \leq \|x\|_1.$$
  As a result, $(L^2 (\bh), \langle \cdot \rangle_2)$ is a Hilbert space.

\begin{theorem}[Hölder's inequality]
  For all $x, y \in L^2 (\bh)$ we have 
  $$\| xy\|_1 \leq \| x\|_2 \| y\|_2.$$
\end{theorem}

\begin{myproof}
  Let $xy = v|xy|$ be the polar decomposition of $xy$. Then 
  \begin{align*}
    \| xy\|_1 &= \trace |xy| = \trace (v^* xy)\\
    &= |\langle y, x^* v \rangle_2| \leq \| x^* v\|_2 \| y\|_2\\
    &\leq \| x^*\|_2 \| v\| \|y\|_2 \leq \| x\|_2 \cdot \|y\|_2. \qedhere
  \end{align*}
\end{myproof}

\subsection{Hilbert--Schmidt integral operators}

Let $(X, \mu)$ be a $\sigma$-finite measure space.
This means that $X$ is a countable union of finite-measure sets:
$$X = \bigcup_{j = 1} ^\infty A_j,\quad \mu(A_j) < \infty.$$
For $K \in L^2 (X \times X, \mu \times \mu)$, we can define a Hilbert--Schmidt integral operator 
with kernel $K$:
$$T_K : L^2 (X, \mu) \to L^2 (X, \mu),\quad f \mapsto \left(y \mapsto \int_X K(x, y) f(y)\, d\mu (x)\right).$$
Suppose $(\varphi_\alpha)_{\alpha}$ is an ONB for $L^2 (K, \mu)$.
By Fubini, $\left(\overline{\varphi_a (x)} \varphi_\beta (y)\right)_{\alpha, \beta}$
is an orthonormal basis for $L^2 (X \times X, \mu \times \mu)$.
Since $K \in L^2 (X \times X, \mu \times \mu)$, there exist $c_{ij} \in \C$ such that 
$$K(x, y) = \sum_{i,j} c_{ij} \overline{\varphi_i (x)} \varphi_j (y),\quad \| K\|^2 _{L^2 (X \times X)} = \sum |c_{ij}|^2 < \infty.$$
We show that $T_K$ is well-defined: for $f \in L^2 (X, \mu)$, we have $T_K f \in L^2 (X,\mu)$.
Indeed,
$$T_k f (y) = \sum_{i, j} c_{ij} \langle f, \varphi_i \rangle \varphi_j (y),$$
which implies 
\begin{align*}
  \| T_K f\|_{L^2 (X)} ^2 &\leq \sum_{i, j} |c_{ij}|^2 |\langle f, \varphi_j\rangle|^2 \| \varphi_j\|^2 _{L^2 (X)}\\
  &\leq \| f\|_{L^2} ^2 \sum_{i, j} |c_{ij}|^2 \| \varphi\|^2 _{L^2} \| \varphi_j\|^2 _{L^2}\\
  &= \| f\|^2 _{L^2} \sum|c_{ij}|^2\\
  &= \| f\|^2 _{L^2} \|K\|^2 _{L^2 (X \times X)}
\end{align*}
and finally $\| T_K\| \leq \|K\|_{L^2}$.
We claim that $T_K ^* : L^2 (X, \mu) \to L^2 (X, \mu)$ is the integral operator with kernel 
$$K^* (y, x) := \overline{K(x, y)}.$$
Indeed, 
\begin{align*}
  \langle T_K f, g \rangle &= \int_Y \left(\int_X K(x, y) f(x) \, d\mu (x)\right) \cdot \overline{g(y)}\, d\mu (y)\\
  &= \int_X f(x) \cdot \left(\overline{\int_Y \overline{K(x, y)} g(y)\, d\mu(y)}\right)\, d\mu (x)\\ 
  &= \langle f, T_{K^*} g \rangle.
\end{align*}

\begin{remark}[Fubini's theorem]
  If $(X, \mu), (Y,\lambda)$ are $\sigma$-finite measure spaces and $\int_{X \times Y} |f|\, d(\mu \times \lambda) (x, y) < \infty$, then 
  $$\int_{X \times Y} f \, d(\mu \times \lambda)(x, y)= \int_Y \left(\int_X f\, d\mu(x)\right)\, d\lambda(y) = \int_X \left(\int_Y f\, d\lambda(y)\right)\, d\mu(x).$$
\end{remark}

\begin{theorem}
  \begin{enumerate}
    \item For $K \in L^2 (X \times X, \mu \times \mu)$ we have $T_K \in L^2 (\mathcal{B} (L^2 (X, \mu)))$.
    \item The mapping $\Phi: K \mapsto T_K$ is a unitary 
    $L^2 (X \times X, \mu \times \mu) \to L^2 (\mathcal{B} (L^2 (X, \mu))).$
  \end{enumerate}
\end{theorem}

\begin{myproof}
  \begin{enumerate}
    \item We will prove that $\| T_K\|_2 = \| K\|_{L^2}$.
    We want to approximate $T_K$ with finite rank operators, so we first approximate $K$:
    $$K (x, y) = \sum_{i, j = 1} ^\infty c_{ij} \overline{\varphi_{i} (x)} \varphi_j (x)$$
    for an orthonormal basis $(\varphi_\alpha)_{\alpha}$ for $L^2 (X, \mu)$.
    For $N \in \N$ let $K_{N} (x, y) = \sum_{i, j} ^N c_{ij} \overline{\varphi_i (x)} \varphi_j (x)$.
    Then 
    $$T_{K_N} f = \sum_{i, j = 1} ^N c_{ij} \langle f, \varphi_i\rangle \varphi_j \in \mathcal{F} (L^2 (X, \mu)).$$
    By the above inequality, 
    $$\| T_K - T_{K_N}\| \leq \| K - K_N\|_{L^2} \to 0,$$
    so $T_K \in \overline{(\mathcal{F}, \| \cdot\|)} = \mathcal{K} (\mathcal{H})$.
    Then $$\| T_K \|^2 _2 = \sum_i \| T_K \varphi_i \|_{L^2} ^2 = \sum_{i, j, k} \| c_{jk} \varphi_j (x) \delta_{ik} \|^2 = \sum |c_{ij}|^2 = \|K \|_{L^2} ^2.$$
    \item It remains to prove surjectivity. Since $\Phi$ is isometric, $\im \Phi$ is closed.
    So it suffices to show that $\im \Phi$ is dense. In particular, we will show that $\im \Phi \supseteq \mathcal{F} (L^2 (X, \mu))$.
    Let $A \in \mathcal{F} (L^2 (X, \mu))$, so $\rank A < \infty$. Let $(\psi_1, \dots, \psi_m)$ be an orthonormal basis for $\im A$.
    Then $A \varphi = c_1 (\varphi) \psi_1 + \dots + c_m (\varphi) \psi_m$ for some bounded linear functionals $c_j$ on $L^2 (X, \mu)$.
    By Riesz, there exist $\mu_j \in L^2 (X, \mu)$ such that $c_j (\varphi) = \langle \varphi, \mu_j \rangle$.
    Hence \begin{equation*}
      A \varphi(x) = \int_X \left(\sum_{j = 1} ^m \psi_j (x) \cdot \overline{\mu_j (y)} \cdot \varphi (y)\right)\, d\mu (y) = T_{\sum_{i = 1} ^m \psi_j (x) \overline{\mu_j (y)}} \in \im \Phi. \qedhere 
    \end{equation*}
  \end{enumerate}
\end{myproof}

\subsection{Hilbert--Schmidt operators $\mathcal{H} \to \mathcal{K}$ and tensor products}

Let $\mathcal{H}, \mathcal{K}$ be Hilbert spaces and $A \in \mathcal{B} (\mathcal{H}, \mathcal{K})$.
We associate to $A$ the map 
$$\widetilde{A} \in \mathcal{B} (\mathcal{H} \oplus \mathcal{K}, \mathcal{H} \oplus \mathcal{K}),\quad \alpha \oplus \beta \mapsto 0 \oplus A \alpha,$$
or in matrix form,
$$\widetilde{A} = \begin{bmatrix}
  0 & 0\\
  A & 0
\end{bmatrix}.$$
We denote the set of Hilbert--Schmidt operators $\mathcal{H} \to \mathcal{K}$ as 
$$HS (\mathcal{H}, \mathcal{K}) := \{A \in \mathcal{B} (\mathcal{H}, \mathcal{K})\ |\ \widetilde{A} \in L^2 (\mathcal{B}(\mathcal{H} \oplus \mathcal{K}))\}.$$ 
It is trivial to show that this coincides with 
$$\{A \in \mathcal{B} (\mathcal{H}, \mathcal{K})\ |\ A^* A \in L^1 (\mathcal{B}(\mathcal{H}))\}.$$
With the usual scalar product $\langle A, B \rangle_{2} = \trace(B^* A)$, this becomes a Hilbert space.

\begin{remark}
  By Riesz's representation theorem, every functional in $\mathcal{H}^*$
  is of the form $\overline{\alpha}: x \mapsto \langle x, \alpha \rangle,$
  where $\alpha \in \mathcal{H}$. 
  This means that we can introduce a scalar product on $\mathcal{H}^*$ by 
  $\langle \overline{\alpha}, \overline{\beta} \rangle_{\mathcal{H}^*} := \langle \beta, \alpha \rangle_{\mathcal{H}}.$
  This scalar product induces the usual operator norm on $\mathcal{H}^*$, so it makes $\mathcal{H}^*$
  into a Hilbert space. 
\end{remark}

\begin{example}
  The dual $\mathcal{H}^*$ is isomorphic as a Hilbert space to $HS(\mathcal{H}, \C)$. To prove this, it's enough to compare the scalar products.
  For any $\overline{\alpha}, \overline{\beta} \in \mathcal{H}^*$, we have 
  \begin{align*}
    \langle \overline{\alpha}, \overline{\beta} \rangle &= \trace (\overline{\beta}^* \overline{\alpha})\\
    &= \sum_{i \in I} \langle \overline{\alpha} e_i, \overline{\beta} e_i\rangle\\
    &= \sum_{i \in I} \langle e_i, \alpha \rangle \overline{\langle e_i, \beta \rangle}\\
    &= \sum_{i \in I} \langle e_i, \alpha \rangle {\langle \beta, e_i \rangle}\\
    &= \langle \beta, \alpha\rangle = \langle \overline{\alpha}, \overline{\beta} \rangle_{\mathcal{H}^*}.
  \end{align*}
\end{example}

\begin{definition}
  We define the \emph{tensor product of Hilbert spaces} as $\mathcal{H} \overline{\otimes} \mathcal{K} := HS (\mathcal{K}^*, \mathcal{H})$. 
\end{definition}

\begin{remark}
  For any $\alpha \in \mathcal{H}$ and $\beta \in \mathcal{K}$, we define the elementary tensors as 
  $$\alpha \otimes \beta: \mathcal{H}^* \to \mathcal{K},\quad f \mapsto f(\alpha) \beta.$$
  The span of these operators is the usual algebraic tensor product of vector spaces $\mathcal{H} \otimes \mathcal{K}$ and
  consists of all the finite-rank operators in $\mathcal{H} \overline{\otimes} \mathcal{K}$.
\end{remark}

\subsection{Locally convex topologies on $\bh$}

If $\mathcal{H}$ is a Hilbert space, then $(\bh, \| \cdot\|)$
is a Banach algebra with its norm topology.

\begin{definition}
  \begin{enumerate}
    \item The \emph{weak operator topology} (WOT) is given by the seminorms 
    $$T \mapsto |\langle T\alpha, \beta\rangle|,\quad \forall \alpha, \beta \in \mathcal{H}.$$
    \item The \emph{strong operator topology} (SOT) is given by the seminorms 
    $$T \mapsto \|T\alpha\|,\quad \forall \alpha \in \mathcal{H}.$$
  \end{enumerate}
\end{definition}

These topologies are comparable: $\mathrm{WOT} \subseteq \mathrm{SOT} \subseteq \textrm{norm topology}$.
\begin{itemize}
  \item Norm topology has the subbasis $$\{S \in \mathcal{B}(\mathcal{H})\ |\ \| S - T\| < \varepsilon \}$$
for $T \in \mathcal{B}(\mathcal{H})$ and $\varepsilon > 0$. The net $T_i$ converges to $T$ iff $\| T_i - T\|$ converges to $0$.
  \item WOT topology has the subbasis $$\{S \in \mathcal{B}(\mathcal{H})\ |\ \langle (S - T) \alpha, \beta \rangle < \varepsilon \}$$
  for $\alpha, \beta \in \mathcal{H}$, $T \in \mathcal{B}(\mathcal{H})$ and $\varepsilon > 0$. The net $T_i$ converges to $T$ iff $\langle T_i \alpha, \beta \rangle$ converges to $\langle T \alpha, \beta \rangle$ for all $\alpha, \beta$.
  \item SOT topology has the subbasis $$\{S \in \mathcal{B}(\mathcal{H})\ |\ \| (S - T) \alpha\| < \varepsilon \}$$
  for $\alpha \in \mathcal{H}$, $T \in \mathcal{B}(\mathcal{H})$ and $\varepsilon > 0$. The net $T_i$ converges to $T$ iff $\| (T_i - T)\alpha \|$ converges to $0$ for all $\alpha$.
\end{itemize}

\begin{example}
  Let $\mathcal{H} = \ell^2 (\N)$ and denote $T_n = \frac{1}{n} \cdot \id$.
  Then $T_n \to 0$ in the norm topology. Now if we introduce the operator
  $$S (x_1, x_2, \dots) = (0, 0, \dots, 0, x_{n + 1}, x_{n + 2}, \dots),$$
  then $S_n \to 0$ in SOT, but not in norm topology, since $\|S_n\| = 1$.
  Lastly, we define 
  $$W_n (x_1, x_2, \dots) = (0, 0, \dots, x_1, x_2, \dots).$$
  We get that $W_n \to 0$ in WOT, but not in SOT or norm topology.
\end{example}

\begin{example}
  Let $(y_n)_n$ be a countable dense subset of $\mathcal{H} = \ell^2$.
  Consider the following two metrics on $(\bh)_1$:
  $$d_S (A, B) = \sum_{n = 1} ^\infty \frac{1}{2^n} \|(A - B) {y_n}\|,\quad d_W (A, B) = \sum_{n = 1} ^\infty \frac{1}{2^n} \left|\langle(A - B) {y_n} , y_n\rangle\right|.$$
  Then $d_S$ induces SOT and $d_W$ induces WOT on $(\bh)_1$.
\end{example}

\begin{example}
  The multiplication 
  $$\bh \times \bh \to \bh,\quad (A, B) \mapsto A \cdot B$$
  is not jointly continuous with regards to SOT or WOT. Indeed, if $S: \ell^2 \to \ell^2$ is the right shift (and $S^*$ the left shift),
  then $S^n \to 0$ and $(S^*)^n \to 0$ in SOT and WOT, but $(S^*)^n S^n = I$. However,
  multiplication is WOT- and SOT-continuous in each factor separately.
  Suppose that $(x_\alpha)_\alpha \to x$ in WOT and $y \in \bh$.
  Then for each $v, w \in \mathcal{H}$, we have 
  $$|\langle x_\alpha yv - xyv, w \rangle | \to 0,$$
  since $x_\alpha \to x$ in WOT. Similarly,
  $$|\langle y x_\alpha v - yxv, w \rangle | = |\langle  x_\alpha v - xv, y^*w \rangle | \to 0,$$
  which implies $x_\alpha y \to xy$ and $y x_\alpha \to yx$ in WOT.
  Similarly, if $(x_\alpha)_\alpha \to x$ in SOT and $y \in \bh$,
  then for each $v \in \mathcal{H}$ we have 
  $$\| (x_\alpha - x)yv\| \rightarrow 0,\quad \| y(x_\alpha - x)v\| \rightarrow 0,$$
  so $x_\alpha y \to xy$ and $y x_\alpha \to yx$ in SOT.
\end{example}

\begin{example}
  The adjoint is isometric in the norm topology. It is also continuous in WOT:
  $$|\langle x^* v - y^* v, w \rangle| < \varepsilon \Leftrightarrow |\langle xw - yw, v\rangle| < \varepsilon.$$
  However, it is not continuous with respect to SOT. If $(e_n)_n$ is an ONB for $\mathcal{H}$, consider $e_1 \otimes \overline{e_n}$.
  Then for every $x \in \mathcal{H}$, we have 
  $$\|(e_1 \otimes \overline{e_n}) x \| = |\langle x, e_n \rangle | \xrightarrow[n \to \infty]{} 0,$$
  so $e_1 \otimes \overline{e_n} \to 0$ in SOT. However,
  $$\| (e_1 \otimes \overline{e_n})^* x\| = \| (e_n \otimes \overline{e_1}) x\| = |\langle x, e_1 \rangle |$$
  does not go to $0$ for all $x \in \mathcal{H}$, which proves our statement.
\end{example}

\begin{remark}
  If $T:X\to Y$ is continuous, then $T$ remains continuous if $X$ is given a finer topology or $Y$ is given a coarser topology. 
  But if both topologies are made coarser or both finer, nothing can be said in general. In particular, if $T:X\to X$ is 
  continuous with respect to a given topology on $X$ in both domain and codomain, you cannot generally conclude anything 
  about continuity of $T$ when $X$ is given a finer or coarser topology on both domain and codomain. The previous example illustrates this.
\end{remark}

\begin{lemma}
  Let $\varphi: \bh \to \C$ be linear. The following are equivalent.
  \begin{enumerate}
    \item There exist $v_1, \dots, v_n \in \mathcal{H}$ and $w_1, \dots, w_n \in \mathcal{H}$ such that 
    $$\varphi (T) = \sum_{i = 1} ^n \langle T v_i, w_i \rangle.$$
    \item $\varphi$ is WOT-continuous.
    \item $\varphi$ is SOT-continuous.
  \end{enumerate}
\end{lemma}

\begin{myproof}
  The implications $(1) \Rightarrow (2) \Rightarrow (3)$ are obvious. Let us prove $(3) \Rightarrow (1)$.
  By proposition \ref{prop:1.2}, there exists a $K > 0$ and $v_1, \dots, v_n \in \mathcal{H}$
  such that 
  $$|\varphi(T)|^2 \leq K \cdot \sum_{i = 1} ^n \| Tv_i\|^2.$$
  Define $$\mathcal{H}_0 := \overline{\left\lbrace \bigoplus_{i = 1} ^n Tv_i\ |\ T \in \bh \right\rbrace} \leq \mathcal{H} ^{\bigoplus n}.$$
  The map 
  $$\mathcal{H}_0 \ni \bigoplus_{i = 1} ^n T v_i \mapsto \varphi(T) \in \C$$
  is a well-defined and bounded linear functional, which by continuity extends to $\mathcal{H}_0 \to \C$.
  By Riesz, there exist $w_1, \dots, w_n \in \mathcal{H}$ such that 
  $$\varphi(T) = \sum_{i = 1} ^n \langle T v_i, w_i \rangle.$$
  Recall that $v \otimes \overline{w} \in \mathcal{F} (\mathcal{H})$ and $\trace (v \otimes \overline{w}) = \langle v, w \rangle$, so 
  $$\trace (T (v \otimes \overline{w})) = \langle Tv, w \rangle.$$
  The previous identity is really 
  $$\varphi(T) = \sum_{i = 1} ^n \trace (T (v \otimes \overline{w})) = \trace (T \cdot \sum_{i = 1} ^n v_i \otimes \overline{w}_i).$$
  This means that $\varphi(T) = \trace (T \cdot A)$ for $A \in \mathcal{F} (\mathcal{H})$.
\end{myproof}

\begin{corollary}
  If $K \subseteq \bh$ is convex, then 
  $$\overline{K}^{\textrm{WOT}} = \overline{K}^{\textrm{SOT}}.$$
\end{corollary}

\begin{myproof}
  Consider $\bh$, equipped with WOT topology. This is a LCS, so $\overline{K}^{w, \textrm{WOT}} = \overline{K}^{\textrm{WOT}}$ 
  by theorem \ref{thm:1.1}. Similarly, we have that $\bh$ is a LCS when equipped with SOT topology, so $\overline{K}^{w, \textrm{SOT}} = \overline{K}^{\textrm{SOT}}$.
  Now 
  \begin{align*}
    x \in \overline{K}^{w, \textrm{WOT}} &\Leftrightarrow \textrm{$\exists$ a net $(x_\alpha)_\alpha \subseteq K$, such that $x_\alpha \to x$ WOT-weakly}\\
    &\Leftrightarrow \textrm{$f(x_\alpha) \to f(x)$ for all WOT-continuous functionals $f: \bh \to \C$}\\
    &\Leftrightarrow \textrm{$f(x_\alpha) \to f(x)$ for all SOT-continuous functionals $f: \bh \to \C$}\\
    &\Leftrightarrow \textrm{$\exists$ a net $(x_\alpha)_\alpha \subseteq K$, such that $x_\alpha \to x$ SOT-weakly}\\
    &\Leftrightarrow x \in \overline{K}^{w, \textrm{SOT}}. 
  \end{align*}
  Therefore, $\overline{K}^{w, \textrm{WOT}} = \overline{K}^{w, \textrm{SOT}}$ and we are done.
\end{myproof}

\begin{definition}
  The \emph{$\sigma$-weak operator topology} ($\sigma$-WOT or ultra-weak) is the topology in $\bh$
  given by the seminorms $$x \mapsto \left| \sum_{i = 1} ^\infty \langle x \alpha_i, \alpha_i \rangle\right| $$
  for $\alpha_i \in \mathcal{H}$ with $\sum_{i = 1} ^\infty \| \alpha_i \|^2 < \infty$.
  A subbasis of open sets is thus 
  $$\left\lbrace x \in \bh\ |\ \left| \sum_{i = 1} ^\infty \langle (x - x_0) \alpha_i, \alpha_i \rangle \right| < \varepsilon \right\rbrace$$
  for $\alpha_i \in \mathcal{H}$ with $\varepsilon > 0$, $x_0 \in \bh$ and $\sum_{i = 1} ^\infty \| \alpha_i\|^2 < \infty$.
\end{definition}

\begin{definition}
  The \emph{$\sigma$-strong operator topology} ($\sigma$-SOT or ultra-strong) is the topology in $\bh$
  given by the seminorms $$x \mapsto \left( \sum_{i = 1} ^\infty \| x \alpha_i \|^2\right)^{\frac{1}{2}}$$
  for $\alpha_i \in \mathcal{H}$ with $\sum_{i = 1} ^\infty \| \alpha_i \|^2 < \infty$.
  A subbasis of open sets is thus 
  $$\left\lbrace x \in \bh\ |\ \left( \sum_{i = 1} ^\infty \| (x - x_0) \alpha_i\|^2 \right)^{\frac{1}{2}} < \varepsilon \right\rbrace$$
  for $\alpha_i \in \mathcal{H}$ with $\varepsilon > 0$, $x_0 \in \bh$ and $\sum_{i = 1} ^\infty \| \alpha_i\|^2 < \infty$.
\end{definition}

\begin{remark}
  $\sigma$-WOT can also be given by seminorms 
  $$x \mapsto |\trace (xa) |$$
  for $a \in L^1 (\bh)$ positive.
  Let $(f_i)_i$ be an ONB for $\mathcal{H}$ and define 
  $$b: \mathcal{H} \to \mathcal{H},\quad f_i \mapsto \alpha_i.$$
  Since $\sum_i ^\infty \| \alpha_i \|^2 < \infty$, we can conclude $b \in L^2 (\bh)$.
  Then:
  \begin{align*}
    \sum_i \langle x \alpha_i, \alpha_i \rangle &= \sum_i \langle xb f_i, b f_i\rangle\\
    &= \sum_i \langle b^* x bf_i, f_i\rangle\\
    &= \trace (b^* xb)\\
    &= \trace (xbb^*),
  \end{align*}
  where $a := b b^* \in L^1 (\bh)$. Since $\bh = L^1 (\mathcal{B} (\mathcal{H}))^*$,
  the $\sigma$-WOT is just the weak-$*$ topology (with respect to this pairing).
\end{remark}

\begin{remark}
  The map 
  $$\id \otimes 1: \bh \to \mathcal{B} (\mathcal{H} \overline{\otimes} \ell^2),\quad x \mapsto x \otimes 1$$
  is an isometric *-isomorphism of $C^*$-algebras. It is neither SOT- nor WOT-continuous.
  Despite that, $\sigma$-WOT on $\bh$ is induced by WOT on $\mathcal{B} (\mathcal{H} \overline{\otimes} \ell^2)$ 
  and the $\sigma$-SOT on $\bh$ is induced by SOT on $\mathcal{B} (\mathcal{H} \overline{\otimes} \ell^2)$. 
  Indeed, if $(e_i)_{i \in \N}$ is an ONB for $\ell^2$, define $\alpha := \sum_{i = 1} ^\infty \alpha_i \otimes e_i \in \mathcal{H} \overline{\otimes} \ell^2$.
  Then 
  $$\sum_{i \in \N} \langle x \alpha_i, \alpha_i \rangle_{\mathcal{H}} = \langle (\id \otimes 1)(x) \alpha, \alpha \rangle_{\mathcal{H} \overline{\otimes} \ell^2}$$ 
  and similarly $$\left(\sum_{i \in \N}  \| x \alpha_i \|^2 _{\mathcal{H}}\right)^{\frac{1}{2}} = \| (\id \otimes 1)(x) \alpha \|_{\mathcal{H} \overline{\otimes} \ell^2}$$ 
\end{remark}

\begin{lemma}
  Let $\varphi: \bh \to \C$ be a linear functional operator. Then the following are equivalent.
  \begin{enumerate}
    \item $\exists a \in L^1 (\bh)$ such that $\varphi(x) = \trace (a x),\ \forall x \in \bh$;
    \item $\varphi$ is $\sigma$-WOT continuous;
    \item $\varphi$ is $\sigma$-SOT continuous.
  \end{enumerate}
\end{lemma}

\begin{myproof}
  As previously, the implication $(1) \Rightarrow (2) \Rightarrow (3)$ is obvious.
  Let us prove $(3) \Rightarrow (1)$. Assume $\varphi$ is $\sigma$-SOT continuous.
  By identifying $\bh$ via $\id \otimes 1$ with a subspace in $\mathcal{B} (\mathcal{H} \otimes \ell^2)$,
  $\varphi$ is SOT-continuous on this subspace. By Hahn--Banach, $\varphi$ extends to a SOT-continuous 
  linear functional on $\mathcal{B} (\mathcal{H} \otimes \ell^2)$.
  By the previous lemma, $\exists \alpha_1, \dots, \alpha_n, \beta_1, \dots, \beta_n \in \mathcal{H} \overline{\otimes} \ell^2$.
  $$\varphi(x) = \sum_{i = 1} ^n \langle (\id \otimes 1) (x) \alpha_i, \beta_i \rangle.$$
  With 
  $$\alpha_i \sum_{j = 1} ^\infty \alpha_{ij} \otimes e_j, \quad \sum_j \| \alpha_{ij} \|^2 < \infty$$
  and 
  $$\beta_i \sum_{j = 1} ^\infty \beta_{ij} \otimes e_j, \quad \sum_j \| \beta_{ij} \|^2 < \infty.$$
  Then 
  \begin{align*}
    \varphi(x) &= \sum_{i = 1} ^n \langle (x \otimes 1) \sum_{j = 1} ^\infty \alpha_{ij} \otimes e_j, \sum_{k = 1} ^\infty \beta_{ik} \otimes e_k\rangle\\
    &= \sum_{i = 1} ^n \sum_{j = 1} ^\infty \sum_{k = 1} ^\infty \langle x \alpha_{ij}, \beta_{ik}\rangle \langle e_j, e_k\rangle\\
    &= \sum_{i = 1} ^n \sum_{j = 1} ^\infty \langle x \alpha_{ij}, \beta_{ij}\rangle.
  \end{align*}
  Define $$A_i : \mathcal{H} \to \mathcal{H},\quad A_i f_k = \alpha_{ik}$$
  and $$B_i : \mathcal{H} \to \mathcal{H},\quad B_i f_k = \beta_{ik}$$
  for an orthonormal basis $(f_k)_{k \in \N}$. By assumption, $A_i, B_i \in L^2 (\bh)$.
  As before, this gives $\varphi (x) = \sum_i \trace (B_i ^* x A_i) = \trace (x A_i B_i^*)$.
\end{myproof}

\begin{corollary}\label{cor:4.2}
  The unit disk $(\bh)_1$ is compact with respect to the $\sigma$-WOT topology.
\end{corollary}

\begin{myproof}
  $\sigma$-WOT on $\bh$ is the weak-$*$ topology from $L^1 (\bh)^* = \bh$.
  The statement now follows from Banach--Alaoglu.
\end{myproof}

\begin{corollary}\label{cor:4.3}
  WOT and $\sigma$-WOT topologies agree on bounded subsets $B \subseteq \bh$.
\end{corollary}

\begin{myproof}
  W.l.o.g.~$B = M \cdot (\bh)_1$ for some $M > 0$. Then the identity $(B, \textrm{$\sigma$-WOT}) \to (B, \textrm{WOT})$
  is a continuous map from a Hausdorff compact space (previous corollary) to a Hausdorff space.
  Therefore the identity map is a closed continuous bijection, so it's a homeomorphism.
\end{myproof}

\begin{definition}
  Let $A$ be a vector space and $B \subseteq \mathcal{L} (A, \C)$ a set of some of its linear functionals.
  Then we define $\sigma(A, B)$ as the weakest topology in $A$ such that functionals in $B$ are continuous. 
\end{definition}

\begin{remark}
  $\sigma$-WOT topology is $\sigma(\bh, L^1 (\bh))$.
\end{remark}

\begin{remark}
  Let us define the following topologies on $\bh$.
  \begin{enumerate}
    \item Weak Banach topology is $\sigma(\bh, \bh^*)$.
    \item Ultrastrong-$*$ topology is the weakest topology stronger than $\sigma$-SOT such that $*$ is continuous.
    \item Strong-$*$ topology is generated by seminorms $x \mapsto \|x\alpha\|$ and $x \mapsto \|x^* \alpha\|$ for $\alpha \in \mathcal{H}$.
  \end{enumerate}
\end{remark}

In the end, we get the following diagram that demonstrates which topologies are comparable.

\[\begin{tikzcd}
	& {\textrm{norm}} \\
	{\textrm{weak Banach}} && {\textrm{ultrastrong-$*$}} \\
	& {\textrm{$\sigma$-SOT}} & {\textrm{strong-$*$}} \\
	{\textrm{$\sigma$-WOT}} && {\textrm{SOT}} \\
	& {\textrm{WOT}}
	\arrow[from=1-2, to=2-3]
	\arrow[from=1-2, to=2-1]
	\arrow[from=2-3, to=3-2]
	\arrow[from=2-3, to=3-3]
	\arrow[from=2-1, to=4-1]
	\arrow[from=3-2, to=4-1]
	\arrow[from=3-2, to=4-3]
	\arrow[from=3-3, to=4-3]
	\arrow[from=4-1, to=5-2]
	\arrow[from=4-3, to=5-2]
\end{tikzcd}\]